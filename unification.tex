
\documentclass{amsart}
\usepackage{amssymb,amsmath,latexsym,stmaryrd,mathtools}
\usepackage{cleveref}
\usepackage{mathpartir}
\let\types\vdash % turnstile
\def\cb{\mid} % context break
\def\op{^{\mathrm{op}}}
\def\p{^+} % variances on variables
\def\m{^-}
\let\mypm\pm
\let\mymp\mp
\def\pm{^\mypm}
\def\mp{^\mymp}
\def\jdeq{\equiv}
\def\cat{\;\mathsf{cat}}
\def\type{\;\mathsf{type}}
\def\ctx{\;\mathsf{ctx}}
\let\splits\rightrightarrows
\def\flip#1{#1^*} % reverse the variances of all variables
\def\mor#1{\hom_{#1}}
\def\id{\mathrm{id}}
\def\ec{\cdot} % empty context
\def\psplit{\overset{\mathsf{pair}}{\splits}}
\def\iso{\cong}
\def\tpair#1#2{#1\otimes #2}
\def\cpair#1#2{\langle #1,#2\rangle}
\def\tlet#1,#2:=#3in{\mathsf{let}\; \tpair{#1}{#2} \coloneqq #3 \;\mathsf{in}\;}
\def\clet#1,#2:=#3in{\mathsf{let}\; \cpair{#1}{#2} \coloneqq #3 \;\mathsf{in}\;}
\def\mix#1,#2 with #3 in{\mathsf{mix} {\scriptsize \begin{array}{c} \check{#1} \coloneqq \check{#3} \\ \hat{#2} \coloneqq \hat{#3} \end{array}  }\mathsf{in}\;}
\newcommand{\coend}{\begingroup\textstyle\int\endgroup}
\newcommand{\Set}{\mathrm{Set}}
\def\yielding{\mid}
\def\unif{\doteq}

\begin{document}

This is for a cartesian language.  I'm not sure how quadraticality plays
into the unification algorithm.

First-order terms consist of variables and constants applied to lists of
terms.  A list of terms is either empty ($\cdot$) or a cons.  

\begin{mathpar}
\inferrule{x \in \Gamma}
          {\Gamma \types x}

\inferrule{\Gamma \types \vec{M}}
          {\Gamma \types F(\vec{M})}

\inferrule{ }
          {\Gamma \types \cdot}

\inferrule{\Gamma \types M \\ \Gamma \types \vec{M}}
          {\Gamma \types M,\vec{M}}
\end{mathpar}

A context is $\cdot$ or $\Gamma,x$.  A substitution is 
\begin{mathpar}
\inferrule{ }
          {\Gamma \vdash \cdot : \cdot}

\inferrule{\Gamma \types \theta : \Gamma' \\ 
            \Gamma \types M }
          {\Gamma \types \theta,M/x : \Gamma',x}
\end{mathpar}

Identity and composition are admissible:
\begin{mathpar}
\inferrule{ }
          {\Gamma \vdash 1_\Gamma : \Gamma}

\inferrule{\Gamma \types \theta : \Gamma' \\ 
            \Gamma' \types \theta'' : \Gamma'' }
          {\Gamma \types \theta'' [ \theta' ] : \Gamma''}
\end{mathpar}


The unification judgements have the form 
\begin{itemize}
\item $M \unif N \yielding \theta$ where $\Gamma \types M$ and $\Gamma
  \types N$ and $\Gamma' \types \theta : \Gamma$.  Operationally, we
  think of $M$ and $N$ and $\Gamma$ as inputs and $\Gamma'$ and $\theta$
  as outputs, so unification produces a new context and a substitution
  for the original variables from it.  

\item $\vec{M} \unif \vec{N} \yielding \theta$ where $\Gamma \types
  \vec{M}$ and $\Gamma \types \vec{N}$ and $\Gamma' \types \theta :
  \Gamma$.  
\end{itemize}

Here are the rules (I put in some type annotations that might be helpful):
\begin{mathpar}

\inferrule{ }
           {\cdot \unif \cdot \yielding 1_\Gamma}

\inferrule{ M \unif N \yielding (\Gamma' \types \theta : \Gamma) \\
             \vec{M}[\theta] \unif \vec{N}[\theta] \yielding (\Gamma'' \types \theta' : \Gamma') \\
           }
           {M,\vec{M} \unif N,\vec{N} \yielding \theta[\theta']}

\inferrule{\Gamma \vdash M}
          {x \unif M \yielding (\Gamma \vdash (1_\Gamma,M/x) : \Gamma,x)}


\inferrule{\vec{M} \unif \vec{N} \yielding \theta}
          {F(\vec{M}) \unif F(\vec{N}) \yielding \theta}
\end{mathpar}


\end{document}
