\documentclass{amsart}
\usepackage{amssymb,amsmath,latexsym,stmaryrd}
\usepackage{cleveref}
\usepackage{mathpartir}
\let\types\vdash % turnstile
\def\cb{\mid} % context break
\def\op{^{\mathrm{op}}}
\def\p{^+} % variances on variables
\def\m{^-}
\let\mypm\pm
\let\mymp\mp
\def\pm{^\mypm}
\def\mp{^\mymp}
\def\jdeq{\equiv}
\def\cat{\;\mathsf{cat}}
\def\type{\;\mathsf{type}}
\def\ctx{\;\mathsf{ctx}}
\let\splits\rightrightarrows
\def\flip#1{#1^*} % reverse the variances of all variables
\def\mor#1{\hom_{#1}}
\def\ec{\cdot} % empty context
\def\psplit{\overset{\mathsf{pair}}{\splits}}
\title{A directed type theory for formal category theory}
\author{Dan Licata \and Andreas Nuyts \and Patrick Schultz \and Michael Shulman}
\begin{document}
\maketitle

\section{Introduction}
\label{sec:introduction}

We describe a two-level theory with two kinds of contexts and types.
The first level, whose types we call \textbf{categories}, is a simple linear type theory with an involutive modality, represented judgmentally by assigning variances to variables in the context.
Thus, for instance, a typical term judgment would look like
\[ x:\p A, y:\m B, z :\p C \types t:D \]
As usual, linearity means that all the rules maintain the invariant that each variable in the context is used exactly once in the conclusion.
In particular, there are no contraction and weakening rules; but we do allow an unrestricted (and usually implicit) exchange rule.
We often write $\Psi$ for contexts of category-variables marked with variances.

The second level is a more complicated sort of linear type theory, whose types we call \textbf{types} (or sometimes \textbf{sets} or \textbf{modules}), and that depends ``quadratically'' on the first level.
Its basic type judgment is
\[ \Psi \types M \type \]
That is, each type depends on some collection of category-variables, with variance.

The basic term judgment for types is
\[ \Delta \cb \Gamma \types M \]
Here $\Gamma$ is a context of type-variables and $M$ is a type.
Unlike the first level, we formulate this second level as a sequent calculus, and instead of including terms in the judgments we regard the derivations themselves as the (normal-form) terms; later on this will necessitate adding an equational theory for derivations.
This sequent calculus is also linear, with no contraction or weakening, but with unrestricted exchange.

The interesting thing happens in the context $\Delta$, which is a list of category-variables \emph{without} variances on which the types in $\Gamma$ and $M$ can depend.
This dependence is ``quadratic'' in the following sense: all the rules maintain the invariant that each variable in $\Delta$ appears \emph{twice} in $\Gamma$ and $M$, and that the two occurrences have opposite variance (where the variance of $\Gamma$ is flipped relative to $M$).
In fact, to be more precise, the actual dependence of $\Gamma$ and $M$ is on a category-context \emph{with} variances that is obtained from $\Delta$ by splitting each variable into two with opposite variance.
The implementation will, of course, use de Brujin indices; when writing judgments with named variables, if $x:A$ is a variable in $\Delta$ we write $\hat{x}:\p A$ and $\check{x}:\m A$ for the two corresponding signed variables that must appear in $\Gamma$ and $M$ (exactly once each).
In other words:
\begin{center}
  $\Delta\cb\Gamma\types N$ requires as preconditions that
  $\begin{array}{c}
    \Delta \splits \flip{\Psi_1},\Psi_2 \\
    \Psi_1 \types \Gamma \ctx\\
    \Psi_2 \types M \type
  \end{array}$
\end{center}
Here $\Delta\splits \flip{\Psi_1},\Psi_2$ means that each variable $x$ in the unsigned context $\Delta$ becomes a pair $\hat x, \check x$ of opposite variance in the signed context $\flip{\Psi_1},\Psi_2$, and that the variances of $\Psi_1$ are then flipped.

The most intuitive version of this ``quadraticality'' rule is when $\hat{x}$ appears in $M$ and $\check{x}$ appears in $\Gamma$:
\[ x:A \cb N(\check x) \types M(\hat x) \]
Then we are simply talking about a morphism of types ``over the category $A$'' --- in semantic terms, a natural transformation.
The dual case when $\check{x}$ appears in $M$ and $\hat{x}$ appears in $\Gamma$ simply represents a natural transformation between \emph{contravariant} functors rather than between covariant ones.
With this case in mind, one might say that we resolve the problem of ``dependent linear type theory'' by stipulating that both the context and the conclusion of a ``linearly dependent judgment'' must \emph{separately} depend ``linearly'' on the same variables.

However, it is the additional freedom to allow $\hat{x}$ and $\check{x}$ to \emph{both} appear in $M$ and neither in $\Gamma$, or vice versa, that gives us the ability to express formal versions of nontrivial facts about category theory.
Semantically, these judgments correspond to what are sometimes called \emph{extraordinary natural transformations}, or simply \emph{extranatural transformations}.
For example, each category $A$ will have a type of morphisms that is contravariant in its first variable and covariant in its second:
\[ x:\m A, y:\p A \types \mor A(x,y) \type \]
To express the \emph{composition} of such morphisms then requires an ``extranaturality judgment'':
\[ x:A, y:A, z:A \cb \mor A(\hat x, \check y), \mor A(\hat y, \check z) \types \mor A(\check x, \hat z) \]
The \emph{identity} morphism judgment is similarly extranatural on the other side:
\[ x:A \cb \ec \types \mor A(\check x,\hat x) \]

\section{Basic rules and type formers}
\label{sec:rules}

\subsection{Structural rules}
\label{sec:structural-rules}

The basic structural rules for context formation are unsurprising, given our linearity restriction:
\begin{mathpar}
  \inferrule{ }{\ec \types \ec \ctx} \and
  \inferrule{\Psi_1 \types \Gamma \ctx \\ \Psi_2 \types M \type}{\Psi_1,\Psi_2 \types \Gamma,M \ctx}
\end{mathpar}

For the moment, we are hoping to describe a cut-free sequent calculus with an admissible cut rule, and also admissible substitution for category-variables.
We postpone describing exactly what the cut rule will look like, since it is somewhat complicated, but we can state the intended identity rule at this point:
\begin{mathpar}
  \inferrule{\Psi \types M \type \\ \Delta\psplit \Psi_1,\Psi_2}{\Delta \cb M[\Psi_1/\Psi] \types M[\Psi_2/\Psi]}
\end{mathpar}
Here the splitting $\Delta \psplit \Psi_1,\Psi_2$ is of a special ``paired'' sort: of the two variables $\hat x$ and $\check x$ arising from any variable $x$ in $\Delta$, one must be in $\Psi_1$ and one in $\Psi_2$.
That is, we apply hats and checks in the $M$ in the conclusion to make the variances match up, and then use the ones that are left over for the $M$ in the context.
For example, we might have
\begin{mathpar}
  \inferrule{x:\p A, y:\m B, z:\p C \types M(x,y,z)\type}{x:A, y:B, z:C \cb M(\check x, \hat y, \check z) \types M(\hat x, \check y, \hat z)}
\end{mathpar}

The substitution rule for category-variables will be
\begin{mathpar}
  \inferrule{\Psi \types a:A \\ \Delta',x:A \cb \Gamma \types M \\ \Delta\psplit \Psi,\Psi'}{\Delta',\Delta \cb \Gamma[a/x] \types M[a/x]}
\end{mathpar}
Once again we have a paired splitting, which essentially expresses that the term $a:A$ depends on the variables in $\Delta$ with some variances.
Recalling that a variable $x$ in the category-context of a term-judgment actually appears as two different variables $\hat x$ and $\check x$ in the type context and conclusion, the substitution notation $[a/x]$ should be regarded as a shorthand for $[\hat a/\hat x, \check a/\check x]$.
More precisely, given the term $a:A$ in the signed context $\Psi$, we can make a new term $\hat a$ out of it by putting hats on all the covariant variables and checks on all the contravariant ones, and dually we can make $\check a$; we then substitute these for $\hat x$ and $\check x$ in both $\Gamma$ and $M$.
(Of course, by quadraticality, each of $\hat x$ and $\check x$ appears exactly once in $\Gamma$ and $M$ combined.)

\subsection{Morphism types}
\label{sec:morphism-types}

Now we are ready to start giving the rules for some type formers.
We begin with the morphism types.
The formation rule is straightforward given the expected variance, and incorporates a substitution.
The right rule involves another paired splitting because it also incorporates a substitution.
\begin{mathpar}
  \inferrule{A\cat \\ \Psi_1\types a:A \\ \Psi_2 \types b:A}{\flip{\Psi_1},\Psi_2 \types \mor A(a,b) \type}\and
  \inferrule{\Psi \types a:A \\ \Delta\psplit \Psi,\Psi'}{\Delta \cb \ec \types \mor A(a,a)}\and
\end{mathpar}
The left rule is significantly easier to state if we do not incorporate a substitution; it is an evident ``directed'' analogue of the elimination rule for equality due to Lawvere and Martin-L\"of.
\begin{mathpar}
  \inferrule{\Delta,x:A \cb \Gamma[\hat x/\hat y] \types M[\hat x/\hat y]}{\Delta,x:A,y:A\cb \Gamma,\mor A(\hat x, \check y) \types M}\and
  % \inferrule{\Delta \cb \Gamma[\check y/\check x] \types M[\check y/\check x]}{\Delta,x:A,y:A\cb \Gamma,\mor A(\hat x, \check y) \types M}\and
\end{mathpar}
When we incorporate substitutions for $x$ and $y$, this becomes
\begin{mathpar}
  \inferrule{\Delta_a \psplit \flip{\Psi_a},\Psi_a \\ \Delta_b \psplit \flip{\Psi_b},\Psi_b \\
    \Delta,x:A \splits \flip{\Psi_\Gamma}, \Psi_M \\\\
    \Psi_\Gamma \types \Gamma\ctx \\ \Psi_M \types M\type \\\\
    \Psi_a \types a:A \\ \Psi_b \types b:A \\
    \Delta,x:A \cb \Gamma \types M}
  {\Delta,\Delta_a,\Delta_b\cb \Gamma[a/\check x,b/\hat x],\mor A(a,b) \types M[a/\check x,b/\hat x]}\and
\end{mathpar}
For instance, we can use this rule to define composition of morphisms:
\begin{equation}
  \inferrule{\inferrule{\inferrule
      {x:A \types x:A \\ x:A \psplit \hat x:\p A, \check x:\m A}
      {x:A\cb \ec \types \mor A(\check x, \hat x)}}
    {x:A,y:A \cb \mor A(\hat x, \check y) \types \mor A(\check x, \hat y)}}
  {x:A, y:A, z:A \cb \mor A(\hat x, \check y), \mor A(\hat y, \check z) \types \mor A(\check x, \hat z)}\label{eq:composition}
\end{equation}
Starting from the bottom we have the left rule on $y,z$, then the left rule again on $x,y$, then the right rule on $x$.
Note that ${x:A,y:A \cb \mor A(\hat x, \check y) \types \mor A(\check x, \hat y)}$ is an instance of the identity rule, so if we had that rule postulated we could stop there; but if identity is to be admissible then it would reduce to the above complete derivation.

Similarly, we have the functorial action of any judgment $x:A \types f(x):B$:
\begin{equation}
  \label{eq:functor}
  \inferrule{\inferrule{x:A \types f(x):B \\ x:A \psplit \hat x:\p A, \check x:\m A}
    {x:A \cb \ec \types \mor B(f(\check x),f(\hat x))}}
  {x:A, y:A \cb \mor A(\hat x, \check y) \types \mor B(f(\check x),f(\hat y))}
\end{equation}

We may also want to consider ``one-sided'' left rules for morphism types (directed analogues of the Paulin-Morhing rule for identity types):
\begin{mathpar}
  \inferrule{\Delta_b \psplit \flip{\Psi_b},\Psi_b \\
    \Delta,x:A \splits \flip{\Psi_\Gamma}, \Psi_M, \Psi_a \\ \hat x, \check x \notin \Psi_a \\\\
    \Psi_\Gamma \types \Gamma\ctx \\ \Psi_M \types M\type \\\\
    \Psi_a \types a:A \\ \Psi_b \types b:A \\
    \Delta,x:A \cb \Gamma \types M}
  {\Delta,\Delta_b\cb \Gamma[b/\hat x],\mor A(a,b) \types M[b/\hat x]}
  \and
  \inferrule{\Delta_a \psplit \flip{\Psi_a},\Psi_a\\
    \Delta,x:A \splits \flip{\Psi_\Gamma}, \Psi_M, \flip{\Psi_b} \\ \hat x, \check x \notin \Psi_b \\\\
    \Psi_\Gamma \types \Gamma\ctx \\ \Psi_M \types M\type \\\\
    \Psi_a \types a:A \\ \Psi_b \types b:A \\
    \Delta,x:A \cb \Gamma \types M}
  {\Delta,\Delta_a\cb \Gamma[a/\check x],\mor A(a,b) \types M[a/\check x]}\and
\end{mathpar}
Either of these should straightforwardly imply the two-sided rule, just as the Paulin-Morhing rule implies the Martin-L\"of one.
The converse implication in full dependent type theory requires a universe, so we probably don't expect it to hold here.
I think we will probably want the one-sided rules.
Note that they can be regarded as the two Yoneda lemmas, one for covariant functors and one for contravariant functors.

With morphism types, we can make the categories into a 2-category: the morphisms are judgments $x:A \types b:B$, and the 2-cells from $b$ to $b'$ are the judgments $x:A \cb \ec \types \mor B(b,b')$.
The identity 2-cell is the $\mor B$-right rule, while the composite of 2-cells (along a morphism) can be obtained from~\eqref{eq:composition} once we have a cut rule.
Prewhiskering of a 2-cell is a substitution, while postwhiskering is a cut with~\eqref{eq:functor}.

\subsection{Tensor types}
\label{sec:tensor-types}

The formation rule for the tensor type:

\begin{mathpar}
  \inferrule{\Psi_M, \Psi \types M\type \\
    \Psi_N, \flip{\Psi} \types N\type}
  {\Psi_M, \Psi_N \types M \otimes_\Psi N\type}\and
\end{mathpar}

The left rule

\begin{mathpar}
  \inferrule{\Delta \splits \flip{\Psi_\Gamma},\flip{\Psi_M},\flip{\Psi_N},\Psi_C \\
    \Delta' \psplit \flip{\Psi},\Psi \\\\
    \Psi_\Gamma \types \Gamma\ctx \\ \Psi_M,\Psi \types M\type \\
    \Psi_N,\flip{\Psi} \types N\type \\ \Psi_C \types C\type \\\\
    \Delta, \Delta' \cb \Gamma,M,N \types C}
  {\Delta \cb \Gamma, M\otimes_\Psi N \types C}\and
\end{mathpar}

and right rule

\begin{mathpar}
  \inferrule{\Delta_a \psplit \flip{\Psi_a},\Psi_a \\ \Delta_M \splits \Psi_M,\flip{\Psi_{\Gamma_M}} \\
    \Delta_N \splits \Psi_N,\flip{\Psi_{\Gamma_N}} \\\\
    \Psi_a \types a:A \\
    \Psi_{\Gamma_M} \types \Gamma_M\ctx \\ \Psi_{\Gamma_N} \types \Gamma_N\ctx \\\\
    \Psi_M,x:\pm A \types M\type \\ \Psi_N,x:\mp A \types N\type \\\\
    \Delta_M,\Delta_a \cb \Gamma_M \types M[a/x] \\
    \Delta_N,\Delta_a \cb \Gamma_N \types N[a/x]}
  {\Delta_M,\Delta_N,\Delta_a \cb \Gamma_M,\Gamma_N \types M\otimes_{x:A} B}\and
\end{mathpar}

\subsection{Hom types}
\label{sec:hom-types}

\subsection{The type classifier}
\label{sec:type-classifier}


\section{Category formers}
\label{sec:category-formers}

\subsection{Opposites}
\label{sec:opposites}

\subsection{Tensor products}
\label{sec:tensor-products}

\subsection{Exponentials}
\label{sec:exponentials}

\subsection{Collages}
\label{sec:collages}



\end{document}
