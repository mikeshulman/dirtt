\documentclass{amsart}
\usepackage{amssymb,amsmath,latexsym,stmaryrd,mathtools}
\usepackage{cleveref}
\usepackage{mathpartir}
\let\types\vdash % turnstile
\def\cb{\mid} % context break
\def\op{^{\mathrm{op}}}
\def\p{^+} % variances on variables
\def\m{^-}
\let\mypm\pm
\let\mymp\mp
\def\pm{^\mypm}
\def\mp{^\mymp}
\def\jdeq{\equiv}
\def\cat{\;\mathsf{cat}}
\def\type{\;\mathsf{type}}
\def\ctx{\;\mathsf{ctx}}
\let\splits\rightrightarrows
\def\flip#1{#1^*} % reverse the variances of all variables
\def\mor#1{\hom_{#1}}
\def\id{\mathrm{id}}
\def\ec{\cdot} % empty context
\def\psplit{\overset{\mathsf{pair}}{\splits}}
\def\iso{\cong}
\def\tpair#1#2{#1\otimes #2}
\def\cpair#1#2{\langle #1,#2\rangle}
\def\tlet#1,#2:=#3in{\mathsf{let}\; \tpair{#1}{#2} \coloneqq #3 \;\mathsf{in}\;}
\def\clet#1,#2:=#3in{\mathsf{let}\; \cpair{#1}{#2} \coloneqq #3 \;\mathsf{in}\;}
\def\mix#1,#2 with #3 in{\mathsf{mix} {\scriptsize \begin{array}{c} \check{#1} \coloneqq \check{#3} \\ \hat{#2} \coloneqq \hat{#3} \end{array}  }\mathsf{in}\;}
\newcommand{\coend}{\begingroup\textstyle\int\endgroup}
\newcommand{\Set}{\mathrm{Set}}

\newcommand{\catctx}{\;\mathsf{catctx}}
\newcommand{\sdiag}{\;\mathsf{sdiag}}
\newcommand{\strs}[1]{\mathfrak{#1}}
\newcommand{\strto}{\curvearrowright}
\newcommand{\introcoend}{\mathsf{incoend}}
\newcommand{\introend}{\mathsf{inend}}
\newcommand{\letexp}[3]{\left(#1\mathrel{\reflectbox{$\mapsto$}}#2\right).#3}

\title{A directed type theory for formal category theory}
\author{Dan Licata \and Andreas Nuyts \and Patrick Schultz \and Michael Shulman}
\begin{document}
\maketitle

\section{Category contexts and string diagrams}
\begin{mathpar}
	\inferrule{
	}{
		\cdot \catctx
	}\and
	\inferrule{
		\Psi \catctx \\
		A \cat
	}{
		\Psi, x :\p A \catctx
	}\and
	\inferrule{
		\Psi \catctx \\
		A \cat
	}{
		\Psi, x :\m A \catctx
	}\and
\end{mathpar}
\begin{mathpar}
	\inferrule{
	}{
		\cdot \sdiag
	}\and
	\inferrule{
		\strs S \sdiag \\
		A \cat
	}{
		\strs S, x :\m A, y:\p A, x \strto y \sdiag
	}\and
\end{mathpar}
We consider both category contexts and string diagrams as sets, rather than lists. We use letters such as $\Upsilon, \Psi, \Phi, \Omega$ for category contexts; $\Gamma, \Delta, \Theta$ for type contexts; Gothic capital letters such as $\strs S$ for string diagrams, and Gothic lower case letters such as $\strs s$ for sets of strings.


We may abbreviate $(\strs S, x :\m A, y \p A, x \strto y)$ as $(\strs S, x \strto y : A)$. If $\Psi' \cong \Psi$, we write $\Psi' \strto \Psi$ to mean $(\flip{{\Psi'}}, \Psi, \strs s)$ where $\strs s$ couples every variable from $\Psi'$ to the corresponding one from $\Psi$.

\section{Structural rules}
\begin{mathpar}
	\inferrule{
		\Psi_\Gamma \vdash \Gamma \ctx \\
		\Psi_M \vdash M \type
	}{
		\Psi_\Gamma, \Psi_M \vdash \Gamma, m : M \ctx
	}\and
	\inferrule{
		\Psi \vdash M \type \\
		\rho : \Psi' \cong \Psi
	}{
		\Psi' \strto \Psi \cb m : M[\rho] \vdash m : M
	}\and
\end{mathpar}
The substitution rule for category variables will be:
\begin{mathpar}
	\inferrule{
		\Psi \vdash a : A \\
		\strs S, x \strto y : A \cb \Gamma \vdash m : M \\
		\rho : \Psi' \cong \Psi
	}{
		\strs S, \Psi' \strto \Psi \cb \Gamma[a[\rho]/x, a/y] \vdash m[a[\rho]/x, a/y] : M[a[\rho]/x, a/y]
	}\and
\end{mathpar}
The substitution rule for type variables will be:
\begin{mathpar}
	\inferrule{
		\Psi_\Gamma \vdash \Gamma \ctx \\
		\Psi_\Delta \vdash \Delta \ctx \\
		\Psi_M \vdash M \type \\
		\Psi_N \vdash N \type \\
		\flip{\Psi_\Gamma}, \Psi_M, \strs s \cb \Gamma \vdash m_0 : M \\
		\flip{\Psi_\Delta}, \flip{\Psi_M}, \Psi_N, \strs t \cb \Delta, m : M \vdash n_0 : N \\
		\text{$\strs s$ and $\strs t$ compose loop-freely at $\Psi_M$}
	}{
		\flip{\Psi_\Gamma}, \flip{\Psi_\Delta}, \Psi_N, \strs s \circ_{\Psi_M} \strs t \cb \Gamma, \Delta \vdash n_0[m_0/m] : N
	}\and
\end{mathpar}
Note that variables have disappeared here. For that reason, I (Andreas) think it may be best (and feasible) to have terms not referring to category variables.

\section{Morphism types}
Formation rule:
\begin{mathpar}
	\inferrule{
		A \cat \\
		\Phi \vdash a : A \\
		\Psi \vdash b : A
	}{
		\flip\Phi, \Psi \vdash \mor A(a, b) \type
	}\and
\end{mathpar}
Right rule:
\begin{mathpar}
	\inferrule{
		A \cat \\
		\Psi, \strs s \sdiag \\
		\flip\Psi \vdash \Gamma \ctx
	}{
		\Psi, \strs s, x \strto y : A \cb \Gamma \vdash \id : \mor A(x, y)
	}\and
\end{mathpar}
Left rule:
\begin{mathpar}
	\inferrule{
		\Psi, \strs s, w \strto z : A \cb \Gamma \vdash m : M
	}{
		\Psi, \strs s, w \strto x : A, y \strto z : A \cb \Gamma, f : \mor A(x, y) \vdash J(f, m) : M
	}\and
\end{mathpar}
Computation: $J(\id, m) \jdeq m$.

\section{Co-end type}
Formation rule:
\begin{mathpar}
	\inferrule{
		\Psi, x :\m A, y :\p A \vdash M \type
	}{
		\Psi \vdash \coend^{x \strto y : A} M \type
	}\and
\end{mathpar}
Right rule:
\begin{mathpar}
	\inferrule{
		\Psi_\Gamma \vdash \Gamma \ctx \\
		\Psi_M, x :\p A, y :\m A \vdash M \type \\
		(\flip{\Psi_\Gamma}, \Psi_M) = (\Psi, w :\m A, z :\p A) \\
		\Psi, \strs s, w \strto x : A, y \strto z : A \cb \Gamma \vdash m : M
	}{
		\Psi, \strs s, w \strto z : A \cb \Gamma \vdash (x \strto y, m) : \coend^{x \strto y : A} M
	}\and
\end{mathpar}
Note that $m$ does not refer to either $x$ or $y$, so we could as well use a notation such as $\introcoend(m)$. However, this notation looks more pair-like and is possibly more practical from a programming perspective, as it allows the programmer to introduce variable names to keep hole types and contexts readable. We consider $(x \strto y, m)$ and $(x' \strto y', m)$ as equal terms, i.e. syntactic sugar for $\introcoend(m)$.

Left rule:
\begin{mathpar}
	\inferrule{
		\Psi_\Gamma \vdash \Gamma \ctx \\
		\Psi_M, x :\m A, y :\p A \vdash M \type \\
		\Psi_N \vdash N \type \\
		\flip{\Psi_\Gamma}, \flip{\Psi_M}, \Psi_N, \strs s, y \strto x : A \cb \Gamma, m : M \vdash n : N
	}{
		\flip{\Psi_\Gamma}, \flip{\Psi_M}, \Psi_N, \strs s \cb \Gamma, q : \coend^{x \strto y:A} M \vdash \letexp{(x \strto y, m)}{q}{n} : N
	}\and
\end{mathpar}
Computation: $\letexp{(x \strto y, m)}{(x \strto y, m')}{n} \jdeq n[m'/m]$

\section{End type}
Formation rule:
\begin{mathpar}
	\inferrule{
		\Psi, x :\m A, y :\p A \vdash M \type
	}{
		\Psi \vdash \coend_{x \strto y : A} M \type
	}\and
\end{mathpar}
Right rule:
\begin{mathpar}
	\inferrule{
		\Psi_\Gamma \vdash \Gamma \ctx \\
		\Psi_M, x :\m A, y :\p A \vdash M \type \\
		\flip{\Psi_\Gamma}, \Psi_M, \strs s, x \strto y : A \cb \Gamma \vdash m : M
	}{
		\flip{\Psi_\Gamma}, \Psi_M, \strs s \cb \Gamma \vdash \lambda(x \strto y).m : \coend_{x \strto y : A} M
	}\and
\end{mathpar}
An identical remark as for the coend applies here: we might as well use a notation such as $\introend(m)$.

Left rule:
\begin{mathpar}
	\inferrule{
		\Psi_\Gamma \vdash \Gamma \ctx \\
		\Psi_M, x :\p A, y :\m A \vdash M \type \\
		\Psi_N \vdash N \type \\
		(\flip{\Psi_\Gamma}, \flip{\Psi_M}, \Psi_N) = (\Psi, w :\p A, z :\m A) \\
		\Psi, \strs s, z \strto y : A, x \strto w : A \cb \Gamma, m : M \vdash n : N
	}{
		\Psi, \strs s, z \strto w : A \cb \Gamma, f : \coend_{x \strto y : A} M \vdash \letexp{\lambda(x \strto y).m}{f}{n} : N
	}\and
\end{mathpar}
Computation: $\letexp{\introend(m)}{\introend(m')}{n} \jdeq n[m'/m]$.

\end{document}