\documentclass{article}
\usepackage{amssymb,amsmath,latexsym,stmaryrd,mathtools}
\usepackage{cleveref}
\usepackage{mathpartir}
\usepackage[backgroundcolor=white,bordercolor=red]{todonotes}
\newcommand{\todoi}[1]{\todo[inline]{#1}}
\let\types\vdash % turnstile
\def\cb{\mid} % context break
\def\op{^{\mathrm{op}}}
\def\p{^+} % variances on variables
\def\m{^-}
\let\mypm\pm
\let\mymp\mp
\def\pm{^\mypm}
\def\mp{^\mymp}
\def\jdeq{\equiv}
\def\cat{\;\mathsf{cat}}
\def\type{\;\mathsf{type}}
\def\ctx{\;\mathsf{ctx}}
\let\splits\rightrightarrows
\def\flip#1{#1^*} % reverse the variances of all variables
\def\mor#1{\hom_{#1}}
\def\id{\mathrm{id}}
\def\ec{\cdot} % empty context
\def\psplit{\overset{\mathsf{pair}}{\splits}}
\def\iso{\cong}
\def\tpair#1#2{#1\otimes #2}
\def\cpair#1#2{\langle #1,#2\rangle}
\def\tlet#1,#2:=#3in{\mathsf{let}\; \tpair{#1}{#2} \coloneqq #3 \;\mathsf{in}\;}
\def\clet#1,#2:=#3in{\mathsf{let}\; \cpair{#1}{#2} \coloneqq #3 \;\mathsf{in}\;}
\def\mix#1,#2 with #3 in{\mathsf{mix} {\scriptsize \begin{array}{c} \check{#1} \coloneqq \check{#3} \\ \hat{#2} \coloneqq \hat{#3} \end{array}  }\mathsf{in}\;}
\newcommand{\coend}{\begingroup\textstyle\int\endgroup}
\newcommand{\Set}{\mathrm{Set}}

\newcommand{\catctx}{\;\mathsf{catctx}}
\newcommand{\sdiag}{\;\mathsf{sdiag}}
\newcommand{\strs}[1]{\mathfrak{#1}}
\newcommand{\strto}{\curvearrowright}
\newcommand{\strtoco}{\rotatebox{180}{$\curvearrowleft$}}
\newcommand{\introcoend}{\mathsf{incoend}}
\newcommand{\introend}{\mathsf{inend}}
%\newcommand{\letexp}[3]{\left(#1 := #2\right).#3}
\newcommand{\letexp}[3]{\left(#1\mathrel{\reflectbox{$\Rrightarrow$}}#2\right).#3}

\title{A directed type theory for formal category theory}
\author{Dan Licata \and Andreas Nuyts \and Patrick Schultz \and Michael Shulman}
\begin{document}
\maketitle

\section{Introduction}
The goal here is to develop a notation that indicates string connections explicitly, rather than relying on variable naming to decide what variables are paired.

At the level of categories, category contexts and category terms, this system is identical to the one in dirtt.tex.

Term judgements look different. The unsigned category contexts are replaced with syntactic string diagrams, which are a mix of signed category variables and string connections, so that every variable is connected to another one by a string.

Types can depend on category variables. In a term judgement, every variable from the string diagram appears exactly once in xeither the context (with opposite variance) xor the conclusion (with proper variance).

Terms, too depend on category variables. They are always mentioned pairwise, in the form $x \strto y$. If $x \strto y$ appears in a constructor, then $x$ should be a contravariant variable and $y$ a covariant one. If it appears in an eliminator (left rule term), the opposite is the case. We can also have pairings of the form $a[\strs s] \strto a$. Here, $a[\strs s]$ is $a$, with all variables substituted for the ones connected to them.

For certain types (see the co-end type), it happens that a variable that is not in the context, occurs twice in the type with opposite variance. This is ok: its type can be inferred from the variables it is paired with. We will consider such variables bound. This situation will also occur as a result of substitution, and will then be rectified by computation.

Furthermore, since our theory is linear, every type variable will occur precisely once in the term.

\section{Details}
\subsection{Category contexts and string diagrams}
\begin{mathpar}
	\inferrule{
	}{
		\cdot \catctx
	}\and
	\inferrule{
		\Psi \catctx \\
		A \cat
	}{
		\Psi, x :\p A \catctx
	}\and
	\inferrule{
		\Psi \catctx \\
		A \cat
	}{
		\Psi, x :\m A \catctx
	}\and
\end{mathpar}
\begin{mathpar}
	\inferrule{
	}{
		\cdot \sdiag
	}\and
	\inferrule{
		\strs S \sdiag \\
		A \cat
	}{
		\strs S, x :\m A, y:\p A, x \strto y \sdiag
	}\and
\end{mathpar}
We consider both category contexts and string diagrams as sets, rather than lists. We use letters such as $\Psi, \Phi$ for category contexts; $\Gamma, \Delta, \Theta$ for type contexts, and Gothic lower case letters such as $\strs s$ for sets of strings.

We may abbreviate $(x :\m A, y \p A, x \strto y)$ as $(x \strto y : A)$. When we write $(\strs s : \Psi' \strto \Psi)$, we mean to say $(\flip{{\Psi'}}, \Psi, \strs s)$, where we are assuming that $\Psi' \cong_\alpha \Psi$ and that $\strs s$ connects every variable from $\Psi'$ with the corresponding one in $\Psi$.

We may use a set of strings $\strs s$ as a substitution, where we mean to replace every variable with the variable at the other end of the string.

\subsection{Structural rules}
\begin{mathpar}
	\inferrule{
		\Psi_\Gamma \vdash \Gamma \ctx \\
		\Psi_M \vdash M \type
	}{
		\Psi_\Gamma, \Psi_M \vdash \Gamma, m : M \ctx
	}\and
	\inferrule{
		\Psi \vdash M \type
	}{
		\strs s : \Psi' \strto \Psi \cb m : M[\strs s] \vdash m_{M[\strs s] \strto M} : M
	}\and
\end{mathpar}
Note that with this variable rule, a variable is not a well-formed term, although we will substitute them for well-formed terms. This means we are going to need a computation rule for variables.

The substitution rule for category variables will be:
\begin{mathpar}
	\inferrule{
		\Psi \vdash a : A \\
		\Phi, \strs s, x \strto y : A \cb \Gamma \vdash m : M \\
	}{
		\Phi, \strs s, \strs t : \Psi' \strto \Psi \cb \Gamma[a[\strs t]/x, a/y] \vdash m[a[\strs t]/x, a/y] : M[a[\strs t]/x, a/y]
	}\and
\end{mathpar}
Observe that this respects all the necessary invariants: $\Gamma$ and $M$ together, and $m$ separately, use every variable from $\Phi$, as well as $x$ and $y$, precisely once. The category term $a$ uses every variable from $\Psi$ exactly once. Substituting $a[\strs t]$ for $x$ and $a$ for $y$ asserts that all variables from $\Phi$, $\Psi$ and $\Psi'$ are used exactly once in the conclusion's types and once in its term.

The substitution rule for type variables will be:
\begin{mathpar}
	\inferrule{
		\Psi_\Gamma \vdash \Gamma \ctx \\
		\Psi_\Delta \vdash \Delta \ctx \\
		\Psi_M \vdash M \type \\
		\Psi_N \vdash N \type \\
		\flip{\Psi_\Gamma}, \Psi_M, \strs s \cb \Gamma \vdash m_0 : M \\
		\flip{\Psi_\Delta}, \flip{\Psi_M}, \Psi_N, \strs t \cb \Delta, m : M \vdash n_0 : N \\
		\text{$\strs s$ and $\strs t$ compose loop-freely at $\Psi_M$}
	}{
		\flip{\Psi_\Gamma}, \flip{\Psi_\Delta}, \Psi_N, \strs s \circ_{\Psi_M} \strs t \cb \Gamma, \Delta \vdash n_0[m_0/m] : N
	}.
\end{mathpar}
Let us check the number of occurrences of category variables in $n_0[m_0/m]$.
\begin{itemize}
	\item $m_0$ uses $\flip{\Psi_\Gamma}, \Psi_M$,
	\item $n_0$ uses $\flip{\Psi_\Delta}, \flip{\Psi_M}, \Psi_N$.
\end{itemize}
Thus, $n_0[m_0/m]$ uses $\flip{\Psi_\Gamma}, \flip{\Psi_\Delta}, \Psi_N$. On top of that, it uses $\Psi_M$ and $\flip{\Psi_M}$, which have disappeared from the context. Let us see what this means if $n_0$ is just the variable $m$:
\begin{mathpar}
	\inferrule{
		\Psi_\Gamma \vdash \Gamma \ctx \\
		\cdot \vdash \cdot \ctx \\
		\Psi_M \vdash M \type \\
		\Psi_M' \vdash M[\strs t] \type \\
		\flip{\Psi_\Gamma}, \Psi_M, \strs s \cb \Gamma \vdash m_0 : M \\
		\strs t : \flip{\Psi_M} \strto \Psi_M' \cb m : M \vdash m_{M \strto M[\strs t]} : M[\strs t]
	}{
		\flip{\Psi_\Gamma}, \Psi_M', \strs s \circ_{\Psi_M} \strs t \cb \Gamma, \vdash (m_0)_{M \strto M[\strs t]} : N
	}.
\end{mathpar}
This suggests the simple computation rule: $(m_0)_{M \strto M[\strs t]} \jdeq m_0[\strs t]$, which removes all double occurrences of variables from $\Psi_M$. Note that $\strs t$ can be completely inferred from ``$M \strto M[\strs t]$'' because $M$ uses every variable in its context precisely once.

\subsection{Non-mixing tensor types}
\subsubsection{Formation}
\begin{mathpar}
	\inferrule{
		\Phi \vdash S \type \\
		\Psi \vdash T \type
	}{
		\Phi \Psi \vdash S \otimes T \type
	}
\end{mathpar}
\subsubsection{Right rule}
\begin{mathpar}
	\inferrule{
		\Phi, \strs s \cb \Gamma \vdash s : S \\
		\Psi, \strs t \cb \Delta \vdash t : T
	}{
		\Phi, \strs s, \Psi, \strs t \cb \Gamma, \Delta \vdash s \otimes t : S \otimes T
	}
\end{mathpar}
\subsubsection{Left rule}
\begin{mathpar}
	\inferrule{
		\Phi, \strs s \cb \Gamma, s : S, t : T \vdash r : R
	}{
		\Phi, \strs s \cb \Gamma, p : S \otimes T \vdash \letexp{s \otimes t}{p}{r} : R
	}
\end{mathpar}
\subsubsection{$\beta$-rule}
Suppose we have $\Phi_S \vdash S \type$ and $\Phi_T \vdash T \type$. Suppose also that $\strs s$, $\strs t$ and $\strs u$ compose loop-freely We can form two terms of the same premises:
\begin{mathpar}
	%\footnotesize
	\inferrule{
		\inferrule{
			\Phi_\Gamma, \Phi_S, \strs s \cb \Gamma \vdash s_0 : S \\\\
			\Phi_\Delta, \Phi_T, \strs t \cb \Delta \vdash t_0 : T
		}{
			{\begin{pmatrix}
				\Phi_\Gamma, \Phi_S, \strs s, \Phi_\Delta, \Phi_T, \strs t \cb \\
				\Gamma, \Delta \vdash s_0 \otimes t_0 : S \otimes T
			\end{pmatrix}}
		} \\
		\inferrule{
			\Phi_S, \Phi_T, \Psi, \strs u \cb \Theta, s : S, t : T \vdash r_0 : R
		}{
			{\begin{pmatrix}
				\Phi_S, \Phi_T, \Psi, \strs u \cb \\
			\Theta, p : S \otimes T \vdash \letexp{s \otimes t}{p}{r_0} : R
			\end{pmatrix}}
		}
	}{
		\Phi_\Gamma, \Phi_\Delta, \Psi, \strs s \circ_{\Phi_S} \strs u \circ_{\Phi_T} \strs t \cb \Theta, \Gamma, \Delta \vdash \letexp{s \otimes t}{s_0 \otimes t_0}{r_0} : R
	}
\end{mathpar}
On the other hand, we can substitute without using tensor products:
\begin{mathpar}
	\inferrule{
		\Phi_\Gamma, \Phi_S, \strs s \cb \Gamma \vdash s_0 : S \\\\
		\Phi_\Delta, \Phi_T, \strs t \cb \Delta \vdash t_0 : T \\\\
		\Phi_S, \Phi_T, \Psi, \strs u \cb \Theta, s : S, t : T \vdash r_0 : R
	}{
		\Phi_\Gamma, \Phi_\Delta, \Psi, \strs s \circ_{\Phi_S} \strs u \circ_{\Phi_T} \strs t \cb \Theta, \Gamma, \Delta \vdash r_0[s_0/s, t_0/t] : R.
	}
\end{mathpar}
This suggests the $\beta$-rule:
\begin{equation}
	\letexp{s \otimes t}{s_0 \otimes t_0}{r_0} \jdeq r_0[s_0/s, t_0/t],
\end{equation}
as one would expect.

\subsubsection{$\eta$-rule}
Let us first apply right and left rule to variables:
\begin{mathpar}
	\inferrule{
		\inferrule{
			\inferrule{
				\Phi \vdash S \type
			}{
				\strs s : \Phi' \strto \Phi \cb s : S[\strs s] \vdash s_{S[\strs s] \strto S} : S
			} \\
			\inferrule{
				\Psi \vdash T \type
			}{
				\strs t : \Psi' \strto \Phi \cb t : T[\strs t] \vdash t_{T[\strs t] \strto T} : T
			}
		}{
			\strs s : \Phi' \strto \Phi, \strs t : \Psi' \strto \Psi \cb s : S[\strs s], t : T[\strs t] \vdash s_{S[\strs s] \strto S} \otimes t_{T[\strs t] \strto T} : S \otimes T
		}
	}{
		\strs s : \Phi' \strto \Phi, \strs t : \Psi' \strto \Psi \cb p : S[\strs s] \otimes T[\strs t] \vdash \letexp{s \otimes t}{p}{s_{S[\strs s] \strto S} \otimes t_{T[\strs t] \strto T}} : S \otimes T
	}
\end{mathpar}
We can also apply the variable rule to tensor types:
\begin{mathpar}
	\inferrule{
		\inferrule{
				\Phi \vdash S \type \\
				\Psi \vdash T \type
		}{
			\Phi, \Psi \vdash S \otimes T \type
		}
	}{
		\strs u : \Upsilon \strto (\Phi, \Psi) \cb p : (S \otimes T)[\strs u] \vdash p_{(S \otimes T)[\strs u] \strto S \otimes T} : S \otimes T
	}
\end{mathpar}
Note that this notation implies that $\Upsilon \cong (\Phi, \Psi)$, so that $\Upsilon$ decomposes into $\Upsilon \jdeq (\Phi', \Psi')$ and $\strs u$ decomposes into $\strs s : \Phi' \strto \Phi$ and $\strs t : \Psi' \strto \Psi$. Thus, both derivation trees end in judgements of the same signature. We can then have the following $\eta$-rule:
\begin{equation}
	\letexp{s \otimes t}{p}{s_{S[\strs s] \strto S} \otimes t_{T[\strs t] \strto T}} \jdeq p_{S[\strs s] \otimes T[\strs t] \strto S \otimes T}. \label{eq:eta-simple}
\end{equation}
However, we can do better. In the derivation tree above, we can include a substitution between the applications the right and the left rule. We start with the outcome of the right rule:
\begin{mathpar}
	\inferrule{
		\inferrule{
			\strs s : \Phi' \strto \Phi, \strs t : \Psi' \strto \Psi \cb s : S[\strs s], t : T[\strs t] \vdash s_{S[\strs s] \strto S} \otimes t_{T[\strs t] \strto T} : S \otimes T \\\\
			\Phi, \Psi, \Upsilon, \strs u \cb \Gamma, q : S \otimes T \vdash r : R
		}{
			\Phi', \Psi', \Upsilon, \strs s \circ_{\Phi} \strs u \circ_{\Psi} \strs t \cb \Gamma, s : S[\strs s], t : T[\strs t] \vdash r[s_{S[\strs s] \strto S} \otimes t_{T[\strs t] \strto T}/q] : R
		}
	}{
		{\begin{pmatrix}
			\Phi', \Psi', \Upsilon, \strs s \circ_{\Phi} \strs u \circ_{\Psi} \strs t \cb \Gamma, p : S[\strs s] \otimes T[\strs t] \vdash \\
			\letexp{s \otimes t}{p}{r[s_{S[\strs s] \strto S} \otimes t_{T[\strs t] \strto T}/p]} : R
		\end{pmatrix}}
	}
\end{mathpar}
Note that the string-diagrams automatically compose loop-freely, as $\strs s$ and $\strs t$ are simply diagrams of natural transformations.

On the other hand, we can substitute an ordinary variable into the same $r$:
\begin{mathpar}
	\inferrule{
		\strs s : \Phi' \strto \Phi, \strs t : \Psi' \strto \Psi \cb p : S[\strs s] \otimes T[\strs t] \vdash p_{S[\strs s] \otimes T[\strs t] \strto S \otimes T} : S \otimes T \\\\
		\Phi, \Psi, \Upsilon, \strs u \cb \Gamma, q : S \otimes T \vdash r : R
	}{
		\Phi', \Psi', \Upsilon, \strs s \circ_{\Phi} \strs u \circ_{\Psi} \strs t \cb \Gamma, p : S[\strs s] \otimes T[\strs t] \vdash r[p_{S[\strs s] \otimes T[\strs t] \strto S \otimes T}/q] : R
	}
\end{mathpar}
This leads to a more general $\eta$-rule:
\begin{equation}
	r[p_{S[\strs s] \otimes T[\strs t] \strto S \otimes T}/q] \jdeq \letexp{s \otimes t}{p}{r[s_{S[\strs s] \strto S} \otimes t_{T[\strs t] \strto T}/p]}. \label{eq:eta}
\end{equation}
The simpler one is derivable from this one, taking $r$ a simple variable ($r \jdeq q_{S \otimes T \strto S[\strs s'] \otimes T[\strs t']}$), and using computation of variables:
\begin{align*}
	\text{LHS}
	&\jdeq (p_{S[\strs s] \otimes T[\strs t] \strto S \otimes T})_{S \otimes T \strto S[\strs s'] \otimes T[\strs t']} \\
	&\jdeq p_{S[\strs s] \otimes T[\strs t] \strto S[\strs s'] \otimes T[\strs t']}, \\
	\text{RHS}
	&\jdeq \letexp{s \otimes t}{p}{(s_{S[\strs s] \strto S} \otimes t_{T[\strs t] \strto T})_{S \otimes T \strto S[\strs s'] \otimes T[\strs t']}} \\
	&\jdeq \letexp{s \otimes t}{p}{(s_{S[\strs s] \strto S[\strs s']} \otimes t_{T[\strs t] \strto T[\strs t']})}
\end{align*}
It only remains to observe that any $S$ and $T$ can be written as $S'[\strs s']$ and $T'[\strs t']$ where $S'$ and $T'$ are the same types in an $\alpha$-equivalent context.

By applying \eqref{eq:eta-simple} to the left of \eqref{eq:eta}, we obtain:
\begin{equation}
	r[\letexp{s \otimes t}{p}{s_{S[\strs s] \strto S} \otimes t_{T[\strs t] \strto T}}/q] \jdeq \letexp{s \otimes t}{p}{r[s_{S[\strs s] \strto S} \otimes t_{T[\strs t] \strto T}/p]}. \label{eq:eta-commut}
\end{equation}
In other words: let-binders commute.

\subsection{Morphism types}
\subsubsection{Formation}
\begin{mathpar}
	\inferrule{
		A \cat \\
		\Phi \vdash a : A \\
		\Psi \vdash b : A
	}{
		\flip\Phi, \Psi \vdash \mor A(a, b) \type
	}\and
\end{mathpar}
\subsubsection{Right rule}
\begin{mathpar}
	\inferrule{
		A \cat
	}{
		x \strto y : A \cb \cdot \vdash \id(x \strto y) : \mor A(x, y)
	}\and
\end{mathpar}
Here, we are using $x$ contravariantly and $y$ covariantly in $\id(x \strto y)$, because this is a constructor.

Right rule with built-in substitution:
\begin{mathpar}
	\inferrule{
		A \cat \\
		\Phi \vdash a : A
	}{
		\strs s : \Phi' \strto \Phi \cb \cdot \vdash \id(a[\strs s] \strto a) : \mor A(a[\strs s], a)
	}\and
\end{mathpar}
\subsubsection{Left rule}
\begin{mathpar}
	\inferrule{
		\Psi, \strs s, w \strto z : A \cb \Gamma \vdash m : M
	}{
		\Psi, \strs s, w \strto x : A, y \strto z : A \cb \Gamma, f : \mor A(x, y) \vdash \letexp{\id(x \strto y)}{f}{m} : M
	}\and
\end{mathpar}
Here we are using $x$ covariantly and $y$ contravariantly, because this is an eliminator. Note that $f$ is not a well-formed term: it is a variable. When we plug a term into it, it will mention $x$ contra- and $y$ covariantly, while both will have disappeared from the context. Computation will resolve this.

With built-in substitution, we can consider source-based, target-based and two-sided elimination. The two-sided one can be derived from either based one, and the target-based one should be derivable from the morphism type of the opposite category. So we just give the source-based one:
\begin{mathpar}
	\inferrule{
		\Phi \vdash b : A \\
		\Psi, \strs s \cb \Gamma[a/z] \vdash m : M[a/z]
	}{
		\Psi, \strs s, \strs t : \Phi' \strto \Phi \cb \Gamma[b/z], f : \mor A(a, b[\strs t]) \vdash \letexp{\id(a \strto b[\strs t])}{f}{m} : M[b/z]
	}\and
\end{mathpar}
The two-sided one is probably more natural, so ideally we could derive the based ones from it.

\subsubsection{$\beta$-rule}
As another demonstration of substitution, let us plug identity into the left rule:
\begin{mathpar}
	\inferrule{
		x \strto y : A \cb \cdot \vdash \id(x \strto y) : \mor{A}(x, y) \\
		\Psi, \strs s, w \strto x : A, y \strto z : A \cb \Gamma, f : \mor A(x, y) \vdash \letexp{\id(x \strto y)}{f}{n} : N
	}{
		\Psi, \strs s, w \strto z : A \cb \Gamma \vdash \letexp{\id(x \strto y)}{\id(x \strto y)}{n} : N
	}.
\end{mathpar}
This suggests the $\beta$-rule 
\[
	\letexp{\id(a \strto b)}{\id(a \strto b)}{m} \jdeq m.
\]

\subsubsection{$\eta$-rule}
We start with the less-general variant. Using right and left rule, we get:
\begin{mathpar}
	\small
		\inferrule{
			\inferrule{
				A \cat
			}{
				w \strto z : A \cb \cdot \vdash \id(w \strto z) : \mor A(w, z)
			}
		}{
			w \strto x : A, y \strto z : A \cb f : \mor A(x, y) \vdash \letexp{\id(x \strto y)}{f}{\id(w \strto z)} : \mor A(w, z)
		}
\end{mathpar}
Using the variable rule, we get:
\begin{mathpar}
	\inferrule{
		\inferrule{
			A \cat
		}{
			w \strto z : A \vdash \mor A(w, z) \type
		}
	}{
		w \strto x : A, y \strto z : A \cb f : \mor A(x, y) \vdash f_{\mor A(x, y) \strto \mor A(w, z)} : \mor A(w, z)
	}
\end{mathpar}
Thus, the simple $\eta$-rule becomes:
\begin{equation}
	f_{\mor A(x, y) \strto \mor A(w, z)} \jdeq \letexp{\id(x \strto y)}{f}{\id(w \strto z)}.
\end{equation}
We can now proceed to the more general one. Note that, in the string diagram of $r$, it is not a bizarre restriction to introduce $w$ and $z$. Indeed, if the string diagrams compose, then the diagram of $r$ cannot connect $x$ and $y$ to each other. By consequence, it must connect them to some $w$ and $z$.
\begin{mathpar}
	\small
	\inferrule{
		\inferrule{
			x \strto y : A \cb \cdot \vdash \id(x \strto y) : \mor A(x, y) \\\\
			w \strto x : A, y \strto z : A, (\Upsilon, \strs u) \cb \Gamma, g : \mor A(x, y) \vdash r : R
		}{
			w \strto z : A, (\Upsilon, \strs u) \cb \Gamma \vdash r[\id(x \strto y)/f] : R
		}
	}{
		w \strto x' : A, y' \strto z : A, (\Upsilon, \strs u) \cb \Gamma, f: \mor A(x', y') \vdash \letexp{\id(x', y')}{f}{r[\id(x \strto y)/g]} : R
	}
\end{mathpar}
On the other hand, we can substitute an ordinary variable into the same $r$:
\begin{mathpar}
	\small
	\inferrule{
		x \strto x' : A, y' \strto y : A \cb f : \mor A(x', y') \vdash f_{\mor A(x', y') \strto \mor A(x, y)} : \mor A(x, y) \\\\
		w \strto x : A, y \strto z : A, (\Upsilon, \strs u) \cb \Gamma, g : \mor A(x, y) \vdash r : R
	}{
		w \strto x' : A, y' \strto z : A, (\Upsilon, \strs u) \cb \Gamma, f : \mor A(x', y') \vdash r[f_{\mor A(x', y') \strto \mor A(x, y)}/g] : R
	}
\end{mathpar}

This leads to the following $\eta$-rule:
\begin{align*}
	r[f_{\mor A(x', y') \strto \mor A(x, y)}/g] \jdeq \letexp{\id(x' \strto y')}{f}{r[\id(x \strto y)/g]}
\end{align*}
which only makes sense when $f$ and $g$ interrupt the same string, in this case $w \strto z$.

Again, plugging the simple $\eta$-rule into the more general one, we obtain a commutation law:
\begin{equation}
	r[\letexp{\id(x' \strto y')}{f}{\id(x \strto y)}/g] \jdeq \letexp{\id(x' \strto y')}{f}{r[\id(x \strto y)/g]}.
\end{equation}

\subsection{Co-end type}
\subsubsection{Formation rule}
\begin{mathpar}
	\inferrule{
		\Psi, x :\m A, y :\p A \vdash M \type
	}{
		\Psi \vdash \coend^{y \strto x : A} M \type
	}\and
\end{mathpar}
We write $y \strto x$ instead of $x \strto y$ so that currying looks good.

\subsubsection{Right rule}
The following right rule is insufficiently general, but easier to understand.
\begin{mathpar}
	\inferrule{
		\Psi_\Gamma \vdash \Gamma \ctx \\
		\Psi_M, x :\m A, y :\p A \vdash M \type \\
		(\flip{\Psi_\Gamma}, \Psi_M) = (\Psi, w :\p A, z :\m A) \\
		\Psi, \strs s, z \strto y : A, x \strto w : A \cb \Gamma \vdash m : M
	}{
		\Psi, \strs s, z \strto w : A \cb \Gamma \vdash (y \strto x, m) : \coend^{y \strto x : A} M
	}\and
\end{mathpar}
Note that the term $(y \strto x, m)$ mentions $x$ and $y$ once in every variance. (Which, to be honest, is my only reason for writing $y \strto x$ and not $x \strto y$ in the term.) We consider $x$ and $y$ bound.

Note also that this right rule is `non-deterministic': when we compare the context of $(y \strto x, m)$ to the one of $m$, we see that the string $z \strto w$ has been interrupted. However, $\Psi$ may contain other strings of type $A$, and we could instead be interrupting any of those other strings. This is perhaps comparable to what happens in cartesian type theory when you use the projections out of the product type. If $\pi_1 p$ has type $M$, then $p$ has type $M \times N$, where $N$ has to be inferred from the syntax of $p$. Here, similarly, we will have to infer what string has been interrupted, from the syntax of $m$. Hopefully that is possible. 

Sufficiently general right rule:
\begin{mathpar}
	\inferrule{
		\Psi_\Gamma \vdash \Gamma \ctx \\
		\Psi_M, x :\m A, y :\p A \vdash M \type \\
		\Phi_w \vdash a : A \\
		(\flip{\Psi_\Gamma}, \Psi_M) = (\Psi, \flip{\Phi_w}, \Phi_z) \\
		(\Psi, \strs s), \strs u : \Phi_z \strto \Phi_y, \strs t : \Phi_x \strto \Phi_w \cb \Gamma \vdash m : M[a_x/x, a_y/y]
	}{
		(\Psi, \strs s), \strs r : \Phi_z \strto \Phi_w \cb \Gamma \vdash (a_y \strto a_x, m) : \coend^{y \strto x : A} M
	}\and
\end{mathpar}
where
\begin{equation}
	a_w :\jdeq a, \qquad
	a_x :\jdeq a[\strs t], \qquad
	a_y :\jdeq a[\strs u \circ_{\Phi_z} \strs r],
	\qquad a_z :\jdeq a[\strs r].
\end{equation}
Here, the resulting term binds all variables from $\Phi_x$ and $\Phi_y$ by double occurrence.

%The condition $(\flip{\Psi_\Gamma}, \Psi_M) = (\Psi, \flip{\Phi_w}, \Phi_z)$ is awkward. We can remove it:
%\begin{mathpar}
%	\inferrule{
%		\Psi_\Gamma \vdash \Gamma \ctx \\
%		\Psi_M, x :\m A, y :\p A \vdash M \type \\
%		\Phi_w \vdash a : A \\
%		\flip{\Psi_\Gamma}, \Psi_M, \Phi_x, \Phi_y, \strs q \cb \Gamma \vdash m : M[a_x/x, a_y/y] \\
%		\text{$\Phi_x$ and $\Phi_y$ do not share any strings in $\strs q$}
%	}{
%		\flip{\Psi_\Gamma}, \Psi_M, \Phi_x, \Phi_y, \strs q \circ_{\Phi_x, \Phi_y} (\strs p : \Phi_x \strto \Phi_y) \cb \Gamma \vdash (a_y \strto a_x, m) : \coend^{y \strto x : A} M
%	}\and
%\end{mathpar}
%where
%\begin{equation}
%	\Phi_w :\jdeq \Phi_x[\strs q], \qquad \Phi_z :\jdeq \Phi_y[\strs q],
%\end{equation}
%\begin{equation}
%	a_w :\jdeq a, \qquad
%	a_x :\jdeq a_w[\strs q], \qquad
%	a_y :\jdeq a_x[\strs p], \qquad
%	a_z :\jdeq a_y[\strs q].
%\end{equation}

%\begin{mathpar}
%	\inferrule{
%		\Psi_\Gamma \vdash \Gamma \ctx \\
%		\Psi_M, x :\m A, y :\p A \vdash M \type \\
%		\Phi_z \vdash a : A \\
%		\rho : \Phi' \cong \Phi \\
%		(\flip{\Psi_\Gamma}, \Psi_M) = (\Psi, \flip{\Phi_w}, \Phi_z) \\
%		\Psi, \strs s, \Phi \strto \Phi^\diamond, {\Phi'}^\diamond \strto \Phi' \cb \Gamma \vdash m : M[a^\diamond[\rho]/x, a^\diamond/y]
%	}{
%		\Psi, \strs s, \Phi \strto \Phi' \cb \Gamma \vdash (a \strto a[\rho], {\diamond}.m) : \coend^{y \strto x : A} M
%	}\and
%\end{mathpar}
%Same idea, but stated more nicely:
%\begin{mathpar}
%	\inferrule{
%		\Psi_\Gamma \vdash \Gamma \ctx \\
%		\Psi_M, x' :\m A, y' :\p A \vdash M \type \\
%		\Psi_f \vdash f : C \to A \\
%		\rho : \Psi_f' \cong \Psi_f \\
%		(\flip{\Psi_\Gamma}, \Psi_M) = (\Psi, w :\m C, z :\p C) \\
%		\Psi, \strs s, z \strto y : C, x \strto w : C, \Psi_f' \strto \Psi_f \cb \Gamma \vdash m : M[f[\rho]x/x', fy/y']
%	}{
%		\Psi, \strs s, z \strto w : C \cb \Gamma \vdash (a \strto a[\rho_w], m) : \coend^{y \strto x : A} M
%	}\and
%\end{mathpar}
%Note that $m$ does not refer to either $x$ or $y$, so we could as well use a notation such as $\introcoend(m)$. However, this notation looks more pair-like and is possibly more practical from a programming perspective, as it allows the programmer to introduce variable names to keep hole types and contexts readable. We consider $(x \strto y, m)$ and $(x' \strto y', m)$ as equal terms, i.e. syntactic sugar for $\introcoend(m)$.
%
%\textbf{Edit:} This rule probably does not admit substitution. We probably need a rule where you can pair with category terms, rather than category variables, in which case it may become more interesting to use a pairing notation.

\subsubsection{Left rule}
\begin{mathpar}
	\inferrule{
		\Psi_\Gamma \vdash \Gamma \ctx \\
		\Psi_M, x :\m A, y :\p A \vdash M \type \\
		\Psi_N \vdash N \type \\
		(\flip{\Psi_\Gamma}, \flip{\Psi_M}, \Psi_N, \strs s), y \strto x : A \cb \Gamma, m : M \vdash n : N
	}{
		\flip{\Psi_\Gamma}, \flip{\Psi_M}, \Psi_N, \strs s \cb \Gamma, q : \coend^{y \strto x:A} M \vdash \letexp{(y \strto x, m)}{q}{n} : N
	}\and
\end{mathpar}
Here, the resulting term uses $x$ positively through $n$ and negatively in the let-binder, and similarly for $y$. We consider these variables bound. This may seem obvious, as it is a let-binder, but note that in the $J$-binder, they were present in the context.

\subsubsection{$\beta$-rule}
We will plug the right rule into the left rule. We use the same names as when we stated the rules, except for $\Gamma$, $\strs s$ and $m$, which we rename to $\Delta$, $\strs t$ and $\mu$ in the left rule.
\begin{mathpar}
	\inferrule{
		(\flip{\Psi_\Gamma}, \Psi_M) = (\Psi, \flip{\Phi_w}, \Phi_z) \\\\
		(\Psi, \strs s), \strs r : \Phi_z \strto \Phi_w \cb \Gamma \vdash (a_y \strto a_x, m) : \coend^{y \strto x : A} M \\\\
		\flip{\Psi_\Delta}, \flip{\Psi_M}, \Psi_N, \strs t \cb \Delta, q : \coend^{y \strto x:A} M \vdash \letexp{(y \strto x, \mu)}{q}{n} : N
	}{
		\flip{\Psi_\Gamma}, \flip{\Psi_\Delta}, \Psi_N, \circ_{\Psi_M}(\strs r, \strs s, \strs t) \cb \Gamma, \Delta \vdash \letexp{(y \strto x, \mu)}{(a_y \strto a_x, m)}{n} : N
	}
\end{mathpar}
From the same premises, we could have derived
\begin{equation}
	\flip{\Psi_\Gamma}, \flip{\Psi_\Delta}, \Psi_N, \circ_{\Psi_M}(\strs r, \strs s, \strs t) \cb \Gamma, \Delta \vdash n[a_x/x, a_y/y, m/\mu] : N.
\end{equation}
So the $\beta$-rule is:
\begin{equation}
	\letexp{(y \strto x, \mu)}{(a_y \strto a_x, m)}{n} \jdeq n[a_x/x, a_y/y, m/\mu].
\end{equation}
Note that $n[a_x/x, a_y/y, m/\mu]$ still refers to the context $\Psi_M$ twice, and will have to be computed further.

%If we substitute the right rule into the left rule, we obtain an expression of the form $\letexp{(x \strto y, m)}{(a \strto b, m')}{n}$. Remember that $a$ and $b$ bind an entire context of type variables into $m'$. The variables $x$ and $y$ occur again in $n$. We compute this expression to $\letexp{(x \strto y, m)}{(a \strto b, m')}{n} \jdeq n[a/x, b/y, m'/m]$, removing the double occurrences of $x$ and $y$ in the usual way. The other double occurrences may still be there, likely triggering further computation.

\subsubsection{$\eta$-rule}
\paragraph{Substitutionless $\eta$ for substitutionless right rule.}
We first consider the less general variant, where we just apply right and left rules to a variable. The right rule with substitution does not apply here.
\begin{mathpar}
	\footnotesize
	\inferrule{
		\inferrule{
			\inferrule{
				x :\m A, y :\p A, \Psi \vdash T(x, y) \type
			}{
				z \strto y : A, x \strto w : A, \strs r : \Psi' \strto \Psi \cb t : T(w, z)[\strs r] \vdash t_{T(w, z)[\strs r] \strto T(x, y)} : T(x, y)
			}
		}{
			z \strto w : A, \strs r : \Psi' \strto \Psi \cb t : T(w, z)[\strs r] \vdash (y \strto x, t_{T(w, z)[\strs r] \strto T(x, y)}) : \coend^{y \strto x : A} T(x, y)
		}
	}{
		\strs r : \Psi' \strto \Psi \cb q : \coend^{z \strto w : A} T(w, z)[\strs r] \vdash \letexp{(z \strto w, t)}{q}{(y \strto x, t_{T(w, z)[\strs r] \strto T(x, y)})} : \coend^{y \strto x : A} T(x, y)
	}
\end{mathpar}
Note that if we call the term's type $Q$, then up to $\alpha$-equivalence the variable $q$ has type $Q[\strs r]$. Then we also have:
\begin{equation}
	q : \coend^{z \strto w} T(w, z)[\strs r] \vdash q_{Q[\strs r] \strto Q} : \coend^{y \strto x : A} T(x, y).
\end{equation}
This gives us the simplified $\eta$-rule
\begin{equation}
	q_{Q[\strs r] \strto Q} \jdeq \letexp{(z \strto w, t)}{q}{(y \strto x, t_{T(w, z)[\strs r] \strto T(x, y)})}.
\end{equation}

\paragraph{$\eta$ with substitution for substitutionless right rule.} 
We start with the outcome of the right rule, and then apply a substitution before applying the left rule:
\begin{mathpar}
		\footnotesize
	\inferrule{
		\inferrule{
			z \strto w : A, \strs r : \Psi' \strto \Psi \cb t : T(w, z)[\strs r] \vdash (y \strto x, t_{T(w, z)[\strs r] \strto T(x, y)}) : \coend^{y \strto x : A} T(x, y) \\\\
			\Upsilon \cb \Gamma, p : \coend^{y \strto x : A} T(x, y) \vdash r : R
		}{
			\Upsilon, z \strto w : A, \strs r : \Psi' \strto \Psi \cb \Gamma, t : T(w, z)[\strs r] \vdash r[(y \strto x, t_{T(w, z)[\strs r] \strto T(x, y)})/p] : R
		}
	}{
		\Upsilon \cb \Gamma, q : \coend^{z \strto w : A} T(w, z)[\strs r] \vdash \letexp{(z \strto w, t)}{q}{r[(y \strto x, t_{T(w, z)[\strs r] \strto T(x, y)})/p]} : R.
	}
\end{mathpar}
We could have obtained the same by simply substituting $q$ into $r$ (still calling the coend type $Q$):
\begin{equation}
	\Upsilon \cb \Gamma, q : \coend^{z \strto w : A} T(w, z)[\strs r] \vdash r[q_{Q[\strs r] \strto Q}/p] : R.
\end{equation}
Thus, we get a slightly more advanced $\eta$-rule:
\begin{equation}
	r[q_{Q[\strs r] \strto Q}/p] \jdeq \letexp{(z \strto w, t)}{q}{r[(y \strto x, t_{T(w, z)[\strs r] \strto T(x, y)})/p]}.
\end{equation}
Combined with the previous one, we get another commutation law:
\begin{align*}
	&r[\letexp{(z \strto w, t)}{q}{(y \strto x, t_{T(w, z)[\strs r] \strto T(x, y)})}/p] \\
	\jdeq &\letexp{(z \strto w, t)}{q}{r[(y \strto x, t_{T(w, z)[\strs r] \strto T(x, y)})/p]}.
\end{align*}

\paragraph{$\eta$ for right rule with substitution.}
Hmm, how does this work? You cannot apply the left rule to an entire context\ldots Unless you do it variable by variable, but then you get an iterated co-end type on the left.

\subsubsection{Co-Yoneda lemma}
We can now prove the co-Yoneda lemma:
\begin{mathpar}
	\inferrule{
		\inferrule{
			w \strto x : A \cb t : T(w) \vdash t_{T(w) \strto T(x)} : T(x) \\
			y \strto z : A \cb \cdot \vdash \id_{y \strto z} : \mor{A}(y, z)
		}{
			w \strto x : A, y \strto z : A \cb t : T(w) \vdash t_{T(w) \strto T(x)} \otimes \id_{y \strto z} : T(x) \otimes \mor{A}(y, z) 
		}
	}{
		w \strto z : A \cb t : T(w) \vdash (x \strto y, t_{T(w) \strto T(x)} \otimes \id(y \strto z)) : \coend^{x \strto y} T(x) \otimes \mor{A}(y, z)
	}
\end{mathpar}
\begin{mathpar}
	\inferrule{
		\inferrule{
			\inferrule{
				x \strto w \cb t : T(x) \vdash t_{T(x) \strto T(w)} : T(w)
			}{
				x \strto y : A, z \strto w : A \cb t : T(x), f : \mor{A}(y, z) \vdash \letexp{\id_{y \strto z}}{f}{t_{T(x) \strto T(w)}} : T(w)
			}
		}{
			x \strto y : A, z \strto w : A \cb s : T(x) \otimes \mor{A}(y, z) \vdash \\
			 \letexp{t \otimes f}{s}{\letexp{\id_{y \strto z}}{f}{t_{T(x) \strto T(w)}}} : T(w)
		}
	}{
		z \strto w : A \cb q : \coend^{x \strto y} T(x) \otimes \mor{A}(y, z) \vdash \\
		\letexp{(x \strto y, s)}{q}{\letexp{t \otimes f}{s}{\letexp{\id_{y \strto z}}{f}{t_{T(x) \strto T(w)}}}} : T(w)
	}
\end{mathpar}
These are inverses. To plug one into the other, we have to avoid overloading of variables. Therefore, we add a prime to every variable that occurs in the inner term, except for the variable where we connect both string diagrams. Here, that variable is $z$:
\begin{align*}
	&\letexp{(x \strto y, s)}{[(x' \strto y', t_{T(w') \strto T(x')} \otimes \id(y' \strto z))]}{\letexp{t \otimes f}{s}{\letexp{\id_{y \strto z}}{f}{t_{T(x) \strto T(w)}}}} \\
	&\jdeq \letexp{t \otimes f}{[t_{T(w') \strto T(x')} \otimes \id(y' \strto z)]}{\letexp{\id_{y' \strto z}}{f}{t_{T(x') \strto T(w)}}} \\
	&\jdeq \letexp{\id_{y' \strto z}}{\id(y' \strto z)}{[t_{T(w') \strto T(x')}]_{T(x') \strto T(w)}} \\
	&\jdeq [t_{T(w') \strto T(x')}]_{T(x') \strto T(w)} \\
	&\jdeq t_{T(w') \strto T(w)}
\end{align*}

In the other direction, we connect at $w$.
We need to prove ($Q(z)$ abbreviates the co-end type):
\begin{align*}
	&q_{Q(z') \strto Q(z)} \jdeq (x \strto y, [\letexp{(x' \strto y', s)}{q}{\letexp{t \otimes f}{s}{ \\
	&\letexp{\id_{y' \strto z'}}{f}{t_{T(x') \strto T(w)}}}}]_{T(w) \strto T(x)} \otimes \id_{y \strto z})
\end{align*}
Both terms have signature
\begin{equation}
	z' \strto z : A \cb q : \coend^{x' \strto y'} T(x') \otimes \mor A{y', z'} \vdash \coend^{x \strto y} T(x) \otimes \mor A{y, z}.
\end{equation}
We can expand the LHS:
\begin{equation*}
q_{Q(z') \strto Q(z)} \jdeq \letexp{(x' \strto y', s)}{q}{(x \strto y, s_{T(x') \otimes \mor A(y', z') \strto T(x) \otimes \mor A(y, z)})}
\end{equation*}
In the RHS, we can use a commutation law to pull the let-binder to the front:
\begin{align*}
	\text{RHS} \jdeq &\letexp{(x' \strto y', s)}{q}{(x \strto y, [\letexp{t \otimes f}{s}{ \\
	&\letexp{\id_{y' \strto z'}}{f}{t_{T(x') \strto T(w)}}}]_{T(w) \strto T(x)} \otimes \id_{y \strto z})}
\end{align*}
Now both sides start with left and right rules of the co-end type.
Then it remains to be proven that
\begin{align*}
	&s_{T(x') \otimes \mor A(y', z') \strto T(x) \otimes \mor A(y, z)} \\
	\jdeq & [\letexp{t \otimes f}{s}{\letexp{\id_{y' \strto z'}}{f}{t_{T(x') \strto T(w)}}}]_{T(w) \strto T(x)} \otimes \id_{y \strto z}
\end{align*}
These terms both have signature
\begin{equation*}
	z' \strto z :A, x' \strto x :A, y \strto y' : A \cb s : T(x') \otimes \mor A(y', z') \vdash T(x) \otimes \mor A(y, z).
\end{equation*}
To see this, let us descend the derivation tree from this point. The right rule of the co-end type binds $x$ and $y$ and leaves a string $x' \strto y'$. Then the left rule pops this string from the context.

We can expand the LHS:
\begin{align*}
	\text{LHS} &\jdeq s_{T(x') \otimes \mor A(y', z') \strto T(x) \otimes \mor A(y, z)} \\
	&\jdeq \letexp{t \otimes f}{s}{(t_{T(x') \strto T(x)} \otimes f_{\mor A(y', z') \strto \mor A(y, z)})}.
\end{align*}
On the RHS, we first use the computation rule for variables to eliminate $w$. 
Then, we can pull the tensor let-binder to the front. Curiously, we can also pull the hom binder to the front, and then push it back to the other component.
\todoi{This is REALLY bizarre and needs precise justification.}
\begin{align*}
	&[\letexp{t \otimes f}{s}{\letexp{\id_{y' \strto z'}}{f}{t_{T(x') \strto T(w)}}}]_{T(w) \strto T(x)} \otimes \id_{y \strto z} \\
	&\jdeq [\letexp{t \otimes f}{s}{\letexp{\id_{y' \strto z'}}{f}{t_{T(x') \strto T(x)}}}] \otimes \id_{y \strto z} \\
	&\jdeq \letexp{t \otimes f}{s}{\letexp{\id_{y' \strto z'}}{f}{(t_{T(x') \strto T(x)} \otimes \id_{y \strto z})}}  \\
	&\jdeq \letexp{t \otimes f}{s}{(t_{T(x') \strto T(x)} \otimes \letexp{\id_{y' \strto z'}}{f}{\id_{y \strto z}})}
\end{align*}
We now have equal components on the left. On the right, we need to prove
\begin{equation}
	f_{\mor A(y', z') \strto \mor A(y, z)} \jdeq \letexp{\id_{y' \strto z'}}{f}{\id_{y \strto z}},
\end{equation}
of signature
\begin{equation}
	z' \strto z : A, y \strto y' : A \cb f : \mor A(y', z') \vdash \mor A(y, z).
\end{equation}
But this is just an $\eta$-law. $\square$

%In the other direction, we connect at $w$. This direction holds if let-binders commute with everything, which seems legitimate (do we typically have this property in a sequent calculus?), as well as $\eta$-laws:
%\todoi{Nonsense?}
%\begin{align*}
%	&(x \strto y, [\letexp{(x' \strto y', s)}{q}{\letexp{t \otimes f}{s}{\letexp{\id_{y' \strto z'}}{f}{t_{T(x') \strto T(w)}}}}]_{T(w) \strto T(x)} \otimes \id_{y \strto z}) \\
%	&\jdeq (x \strto y, [\letexp{(x' \strto y', s)}{q}{\letexp{t \otimes f}{s}{\letexp{\id_{y' \strto z'}}{f}{t_{T(x') \strto T(x)}}}}] \otimes \id_{y \strto z}) \\
%	&\jdeq \letexp{(x' \strto y', s)}{q}{(x \strto y, [\letexp{t \otimes f}{s}{\letexp{\id_{y' \strto z'}}{f}{t_{T(x') \strto T(x)}}}] \otimes \id(y \strto z))} \\
%	&\jdeq \letexp{(x' \strto y', s)}{q}{(x \strto y, [\letexp{t \otimes f}{s}{\letexp{\id_{y' \strto z'}}{f}{t_{T(x') \strto T(x)}}} \otimes \id(y \strto z)])} \\
%%	&\jdeq \letexp{(x' \strto y', s)}{q}{
%%		\letexp{t \otimes f}{s}{
%%			\letexp{\id_{y' \strto z'}}{f}{
%%				(x \strto y, t_{T(x') \strto T(x)} \otimes \id_{y \strto z})
%%			}
%%		}
%%	}
%\end{align*}

\subsection{End type}
\todoi{Not up-to-date}
Formation rule:
\begin{mathpar}
	\inferrule{
		\Psi, x :\m A, y :\p A \vdash M \type
	}{
		\Psi \vdash \coend_{x \strto y : A} M \type
	}\and
\end{mathpar}
Right rule:
\begin{mathpar}
	\inferrule{
		\Psi_\Gamma \vdash \Gamma \ctx \\
		\Psi_M, x :\m A, y :\p A \vdash M \type \\
		\flip{\Psi_\Gamma}, \Psi_M, \strs s, x \strto y : A \cb \Gamma \vdash m : M
	}{
		\flip{\Psi_\Gamma}, \Psi_M, \strs s \cb \Gamma \vdash \lambda(x \strto y).m : \coend_{x \strto y : A} M
	}\and
\end{mathpar}
An identical remark as for the coend applies here: we might as well use a notation such as $\introend(m)$.

Left rule:
\begin{mathpar}
	\inferrule{
		\Psi_\Gamma \vdash \Gamma \ctx \\
		\Psi_M, x :\p A, y :\m A \vdash M \type \\
		\Psi_N \vdash N \type \\
		(\flip{\Psi_\Gamma}, \flip{\Psi_M}, \Psi_N) = (\Psi, w :\p A, z :\m A) \\
		\Psi, \strs s, z \strto y : A, x \strto w : A \cb \Gamma, m : M \vdash n : N
	}{
		\Psi, \strs s, z \strto w : A \cb \Gamma, f : \coend_{x \strto y : A} M \vdash \letexp{\lambda(x \strto y).m}{f}{n} : N
	}\and
\end{mathpar}
Computation: $\letexp{\introend(m)}{\introend(m')}{n} \jdeq n[m'/m]$.

\end{document}