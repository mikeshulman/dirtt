\documentclass{amsart}

\newif\ifcref\creftrue
\input{decls}
\usepackage{ifmtarg,tikz}
\tikzset{lab/.style={auto,font=\scriptsize}} % arrow labels
\usetikzlibrary{arrows}
\usetikzlibrary{shapes.geometric,shapes.arrows}
\newenvironment{tikzc}[1][]{\begin{center}\begin{tikzpicture}[#1]}{\end{tikzpicture}\end{center}}
\tikzset{>=stealth}
\tikzset{ed/.style={auto,inner sep=0pt,font=\scriptsize}} %edges

\newcommand{\C}{\cC}
\newcommand{\V}{\cV}
\newcommand{\W}{\cW}
\newcommand{\K}{\sK}

\autodefs{\cMod}

\newcommand{\blank}{\mathord{\hspace{1pt}\text{--}\hspace{1pt}}}

\title{Cyclic virtual equipments}
\author{Michael Shulman}
\begin{document}
\maketitle

Let $S$ be the monad on $\mathbf{Cat}$ whose algebras are symmetric strict monoidal categories equipped with a symmetric strict monoidal involution.
Concretely, the objects of $S\C$ are finite lists of objects of $\C$, some of which are marked with a formal $(\blank)\op$, such as $(A,B\op,B,C\op,A)$.
The morphisms of $S\C$ are finite lists of morphisms of $\C$ labeled by permutations which respect the opposites, e.g.\ a morphism $(A,B\op,B,C\op,A) \to (D,B,D\op,C,A\op)$ might be given by the following:
\begin{tikzc}
  \node (A1) at (0,3) {$A$};
  \node (Bo2) at (1,3) {$B\op$};
  \node (B3) at (2,3) {$B$};
  \node (Co4) at (3,3) {$C\op$};
  \node (A5) at (4,3) {$A$};
  \node (D1') at (0,0) {$D$};
  \node (B2') at (1,0) {$B$};
  \node (Do3') at (2,0) {$D\op$};
  \node (C4') at (3,0) {$C$};
  \node (Ao5') at (4,0) {$A\op$};
  \draw[->] (A1) to[out=-90,in=90] node[ed,swap,pos=.2] {$f$} (B2');
  \draw[->] (Bo2) to[out=-90,in=90] node[ed,swap,pos=.1] {$g$} (Ao5');
  \draw[->] (B3) to[out=-90,in=90] node[ed,swap,pos=.9] {$h$} (D1');
  \draw[->] (Co4) to[out=-90,in=90] node[ed,swap,pos=.8] {$k$} (Do3');
  \draw[->] (A5) to[out=-90,in=90] node[ed,pos=.2] {$\ell$} (C4');
\end{tikzc}
Here $f:A\to B$, $g:B\to A$, $h:B\to D$, $k:C\to D$, and $\ell:A\to C$ are morphisms in $\C$.
Note that there can only be a morphism between two lists if they have the same length \emph{and} the same number of opposites.

Since $S\C$ is monoidal with an involution, its objects can be concatenated and oppositized, so for instance
\[(A,B\op)\cdot (C\op,D,A) = (A,B\op,C\op,D,A) \quad\text{and}\quad (B\op,A,A)\op = (B,A\op,A\op).\]
Now suppose $\C$ is a groupoid.
If $\alpha$ is a word in $S\C$ of length $n$, define a \emph{pairing} on $\alpha$ to be a partition of $[n]$ into 2-element subsets (so that in particular $n$ must be even) such that in each pair exactly one of the corresponding objects in $\alpha$ is opposite (so that in particular exactly half of the objects in $\alpha$ must be opposite), together with isomorphisms in $\C$ between each two paired objects.
Now let $H_\C$ denote the \emph{pairing profunctor} from $S\C$ to $S\C$, where $H_\C(\alpha,\beta)$ is the set of pairings on $\alpha\op\cdot\beta$.
Thus we either pair two objects in $\alpha$ or two objects in $\beta$ of which one is opposite, or we pair an object in $\alpha$ with an object in $\beta$ that are both or neither opposite.
For instance, an element of $H_\C((A,B\op,A\op,C),(C,B\op)$ could be drawn like this:
\begin{tikzc}[scale=.7]
  \node (A1) at (0,3) {$A$};
  \node (Bo2) at (0,2) {$B\op$};
  \node (Ao3) at (0,1) {$A\op$};
  \node (C4) at (0,0) {$C$};
  \node (C1') at (3,1) {$C$};
  \node (Bo2') at (3,0) {$B\op$};
  \node (D3') at (3,2) {$D$};
  \node (D4') at (3,3) {$D\op$};
  \draw (A1) to[out=0,in=0] (Ao3);
  \draw (Bo2) to[out=0,in=180] (Bo2');
  \draw (C4) to[out=0,in=180] (C1');
  \draw (D3') to[out=180,in=180,looseness=1.5] (D4');
\end{tikzc}
The actions of $S\C$ are given by rearranging the pairs and composing isomorphisms; we need $\C$ to be a groupoid since we end up having to compose in both directions.

In what follows we will be interested in a particular restriction of the pairing profunctor.
Let $R$ denote the free strict monoidal category monad.
We have a obvious map of monads $R\to S$, and composing it with the multiplication of $S$ we have a functor $\theta : RSX \to SSX \to SX$.
On the other hand, for a set $X$ (regarded as a discrete category) taking the set of objects of $SX$ gives a monad $S_0$ on $\mathbf{Set}$ whose algebras are monoids with an involution.
We again have an obvious map of monads $S_0 \to S$, and an induced composite $\psi : SS_0X \to SSX \to SX$.
Let $H'_X = H_X(\theta,\psi)$, a profunctor from $SS_0X$ to $RSX$; thus we have just decomposed the domain and codomain of $H$ in a particular way.

Now we define a \textbf{node family} to consist of a set $X$, regarded as a discrete category, together with a diagram $Y:SX\to\mathbf{Set}$.
We picture the elements of $Y$ as labels for nodes in directed graphs, equipped with an (ordered) labeling on their edges: the objects labeled with $(\blank)\op$ correspond to edges going out, while those without $(\blank)\op$ are edges going in.
For instance, $M\in Y(A,B\op,B,C\op,A)$ might be drawn like this:
\begin{tikzc}
  \node[rectangle,draw] (M) at (0,0) {$M$};
  \draw[<-] (M) -- node[ed] {$A$} (1,1) node[ed,anchor=north west] {1};
  \draw[->] (M) -- node[ed] {$B$} (1,-1) node[ed,anchor=south west] {2};
  \draw[<-] (M) -- node[ed] {$B$} (-.3,-1) node[ed,anchor=north east] {3};
  \draw[->] (M) -- node[ed] {$C$} (-1,0) node[ed,anchor=east] {4};
  \draw[<-] (M) -- node[ed] {$A$} (-.3,1) node[ed,anchor=south east] {5};
\end{tikzc}
Since $X$ is discrete, the only morphisms in $SX$ are permutations respecting the opposites.
This essentially means we have operations allowing us to renumber the edges of any node.
Note that these actions are not in general free: if for instance $M \in Y(A,A)$ it might, or might not, be the case that $M$ is fixed by switching the two $A$'s.

Given a node family $(X,Y)$, we define a new node family $T(X,Y)$ as follows.
Its underlying set is $S_0X$.
To define the nodes of $T(X,Y)$, first note that the free strict monoidal category monad $R$ can be extended to act on diagrams, giving a diagram $RY : RSX \to \mathbf{Set}$.
Now we tensor this diagram with $H'_X$ over $RSX$, obtaining a diagram $SS_0X \to \mathbf{Set}$; this is the diagram of nodes in $T(X,Y)$.
More concretely, these nodes are obtained as follows:
\begin{enumerate}
\item Consider a finite list of nodes in $Y$
\item Renumber their combined edges arbitrarily (compatibly with how we already know how to renumber the edges of each node individually)
\item Pair up some of their edges that have opposite arity (the loops on the left side of $H$) and perhaps add some new ``free'' edges having no nodes (the loops on the right side of $H$)
\item Decompose the labels on the remaining outer edges into a word of words for $S$, e.g.\ $(A,B\op,C,B,B\op,A\op)$ could become $((A,B\op),(C),(),(B\op,B,A)\op)$.
\end{enumerate}
We call these nodes of $T(X,Y)$ \emph{labeled graphs}.
If the nodes are ``generalized proarrows'', then a labeled graph is a way to ``compose them up'': the pairings indicate where to perform tensor products of functors, the disconnected components of the graph are simply tensored together, and the free edges added on the right of $H$ are hom-functors to also tensor in.

The operation $T$ is almost a monad on the category of node families.
We have a map $(X,Y) \to T(X,Y)$ arising from the units of the monads $S$ and $R$: on nodes it does no renumbering, pairing, or antipairing, and decomposes $(A,B\op,C)$ as $((A),(B)\op,(C))$.
And we have a \emph{partial} map $TT(X,Y)\rightharpoonup T(X,Y)$ defined as follows.

On objects it is simply the multiplication of the monad $S_0$ (and hence is total).
On nodes, what we need is a map
\[H'_{S_0 X} \otimes_{RSS_0X} R(H'_X \otimes_{RSX} RY) \rightharpoonup H'_X \otimes_{RSX} RY\]
lying over the map $SS_0S_0 X \to SS_0 X$ ($S$ of the multiplication of the monad $S_0$).
Now it is a fact that $R$ extended to profunctors is a strong functor, so our desired domain can be identified with
\[ H'_{S_0 X} \otimes_{RSS_0X} RH'_X \otimes_{RRSX} RRY \]
or, looking at the definition of $H'$ in terms of $H$,
\[ {SS_0X}(1,\phi S_0) \odot % \otimes_{S S_0 X}
  H_{S_0 X} \odot %\otimes_{S S_0 X}
  SS_0X(\theta S_0,1) \odot %\otimes_{RSS_0 X}
  RSX(1,R\phi)\odot % \otimes_{RSX}
  R H_X \odot % \otimes_{RSX}
  RSX(R\theta,1) \odot %\otimes_{RRSX}
  RRY \]
Now the following diagram commutes, by naturality:
\begin{equation*}
  \vcenter{\xymatrix{
      RSS_0 \ar[r]^{R\phi}\ar[d]_{\theta S_0} &
      RS\ar[d]^{\theta}\\
      SS_0\ar[r]_{\phi} &
      S.
    }}
\end{equation*}
Therefore, we have a map
\[SS_0X(\theta S_0,1) \odot RSX(1,R\phi) \too SX(1,\phi) \odot SX(\theta,1)
\]
Applying this in the middle of the above long composite, we get
\[ {SS_0X}(1,\phi S_0) \odot
  H_{S_0 X} \odot
  SX(1,\phi) \odot
  SX(\theta,1) \odot
  RH_x \odot
  RSX(R\theta,1) \odot 
  RRY \]
Now we have two cells of profunctors
\begin{equation*}
  \vcenter{\xymatrix{
      RSX\ar[r]|{|}^{R H_X}\ar[d]_{\theta} \ar@{}[dr]|{\Downarrow} &
      RSX \ar[d]^{\theta}\\
      SX\ar[r]|{|}_R &
      SX
    }}\qquad\text{and}\qquad
  \vcenter{\xymatrix{
      SS_0X\ar[r]|{|}^{H_{S_0 X}}\ar[d]_{\phi} \ar@{}[dr]|{\Downarrow} &
      SS_0X \ar[d]^{\phi}\\
      SX\ar[r]|{|}_R &
      SX
    }}
\end{equation*}
The first is obtained by discarding the brackets in a list of pairings, and the second by ``distributing'' a pairing between words to a pairing between their elements.
This gives us maps
\[SX(\theta,1) \odot RH_x \to H_x \odot SX(\theta,1) \]
\[ H_{S_0 X} \odot SX(1,\phi) \to SX(1,\phi) \odot H_X. \]
Applying these in the middle of the long composite, we get
\[ {SS_0X}(1,\phi S_0) \odot
  SX(1,\phi) \odot
  H_{X} \odot
  H_X \odot SX(\theta,1) \odot
  RSX(R\theta,1) \odot 
  RRY. \]
Next we note that $\phi \circ \phi S_0 = \phi \circ S \mu$ and $\theta \circ R\theta = \theta \circ \nu S$, where $\mu$ and $\nu$ are the multiplications of the monads $S_0$ and $R$ respectively.
Thus, this is isomorphic to
\[ {SS_0X}(1,S\mu) \odot
  SX(1,\phi) \odot
  H_{X} \odot
  H_X \odot SX(\theta,1) \odot
  RSX(\nu S,1) \odot 
  RRY. \]
Since we want our map to be over $S\mu$, we can drop the ${SS_0X}(1,S\mu)$ on the left.
On the right, the multiplication $RRY \to RY$ lies over $\nu S$, hence induces a map $RSX(\nu S,1) \odot RRY \to RY$.
Thus, we have gotten to
\[ SX(1,\phi) \odot
  H_{X} \odot
  H_X \odot SX(\theta,1) \odot
  RY. \]
This is almost the diagram of nodes of $T(X,Y)$; it remains only to collapse the two copies of $H_X$.
Here is where the partiality enters.
We define a partial composition law $H\otimes H \rightharpoonup H$ by ``following the paths of pairings'' until they end up on the outside, as below:
 \begin{tikzc}[scale=.7]
  \node (A1) at (0,3) {$A$};
  \node (Bo2) at (0,2) {$B\op$};
  \node (Ao3) at (0,1) {$A\op$};
  \node (C4) at (0,0) {$C$};
  \node (C1') at (3,1) {$C$};
  \node (Bo2') at (3,0) {$B\op$};
  \node (D3') at (3,2) {$C\op$};
  \node (D4') at (3,3) {$C$};
  \node (Do1'') at (6,2) {$C$};
  \node (Bo2'') at (6,1) {$B\op$};
  \draw (A1) to[out=0,in=0] (Ao3);
  \draw (Bo2) to[out=0,in=180] (Bo2');
  \draw (C4) to[out=0,in=180] (C1');
  \draw (D3') to[out=180,in=180,looseness=1.5] (D4');
  \draw (D4') to[out=0,in=180] (Do1'');
  \draw (Bo2') to[out=0,in=180] (Bo2'');
  \draw (C1') to[out=0,in=0,looseness=1.5] (D3');
  \node at (8,1.5) {$\Rightarrow$};
  \begin{scope}[xshift=10cm]
    \node (xA1) at (0,3) {$A$};
    \node (xBo2) at (0,2) {$B\op$};
    \node (xAo3) at (0,1) {$A\op$};
    \node (xC4) at (0,0) {$C$};
    \node (xDo1'') at (3,2) {$C$};
    \node (xBo2'') at (3,1) {$B\op$};
    \draw (xA1) to[out=0,in=0] (xAo3);
    \draw (xBo2) to[out=0,in=180] (xBo2'');
    \draw (xC4) to[out=0,in=180] (xDo1'');
  \end{scope}
\end{tikzc}
However, this composition is only defined when there are no resulting ``loops'' in the middle:
\begin{tikzc}
  \matrix[column sep=1.2cm,row sep=.3cm]{\node (a11) {$A$}; & \node (a21) {$A$}; & \\
    \node (a12) {$C\op$}; & \node (a22) {$D\op$}; & \node (a32) {$A$}; \\
    \node (a13) {$B$}; & \node (a23) {$D$}; & \node (a33) {$B$}; \\
    \node (a14) {$C$}; & \node (a24) {$B$}; & \\
  };
  \draw (a11) -- (a21) to[out=0,in=180] (a32);
  \draw (a13) to[out=0,in=180] (a24);
  \draw (a24) to[out=0,in=180] (a33);
  \draw (a12) to[out=0,in=0] (a14);
  \draw (a22) to[out=0,in=0,looseness=1.5] (a23);
  \draw (a22) to[out=180,in=180,looseness=1.5] (a23);
  \node[red!40,rotate=20] {\LARGE NOT ALLOWED};
\end{tikzc}
In other words, while we allow ourselves to create ``loops'' by pairing up two edges of the same node, or more generally by creating circuits between nodes with the pairings, we do not allow ``free loops'' that contain no nodes at all.

It is easy to see that both composites $T(X,Y) \to TT(X,Y) \rightharpoonup T(X,Y)$ are (total and) the identity.
The associativity square is more tedious but still not difficult.
Moreover, insofar as it makes sense, the ``monad'' $T$ is \emph{cartesian}, enabling us to apply Leinster's machinery of generalized multicategories to it.\footnote{The reason we have to forbid free loops is to obtain both the associativity square and the cartesian property: if we \emph{list} the free loops in some order, then the associativity square will only hold ``up to isomorphism'' since these lists may have to be reordered, whereas if we give them no order at all (such as considering the free abelian group on the set of objects) then cartesianness would fail.}
% TODO: Check those more carefully.

A $T$-multicategory, therefore, will consist of a span $(X_0,Y_0) \leftarrow (X_1,Y_1) \to T(X_0,Y_0)$ of node families with unit and compositional structure.
We deal with the partiality of the monad multiplication by only requiring the multicategory composition to be defined when the multiplication of $T$ is.

We call $X_0$ the \emph{objects} and $X_1$ the \emph{arrows}.
They form an $S_0$-multicategory; this is like an ordinary non-symmetric multicategory, but some of the objects in the domain of an arrow can be labeled ``opposite'', and such opposites ``distribute through'' when morphisms are composed.
Of course, $(X_0,Y_0)$ is a node family; we call the elements of $Y_0$ \emph{modules} and their parametrizing word in $SX_0$ their \emph{boundary}; if $\omega\in SX_0$ we write $\cMod(\omega)$ for the set (and, later, the category) of modules with boundary $\omega$.

It remains to describe $Y_1$, whose elements we call \emph{cells}.
This is a node family over $X_1$, the set of morphisms in the underlying $S_0$-multicategory; thus each of its elements is parametrized by a finite list of arrows, some opposite.
We call this list its \emph{arrow boundary}.
The \emph{target} map of the $T$-multicategory assigns to any cell $\alpha$ a module whose boundary is the target of the arrow boundary of $\alpha$.

Finally, the \emph{source} of a cell $\alpha$ is a labeled graph over $(X_0,Y_0)$ (a diagram of modules that ``could be composed'').
Recalling that the outer edges of a labeled graph are decomposed into a word of words, this must match up with the word obtained from the sources of the arrow boundary of $\alpha$.

Here is a picture of a generic cell $\alpha$ with its source, target, and arrow boundary.
We have not numbered anything on the picture, but technically all the arrows, nodes, and edges appear in ordered lists.
\begin{tikzc}[xscale=.9]
  \node[draw] (Q) at (10,3) {$Q$};
  \draw[<-] (Q) -- node[ed,swap] {$A$} +(1.5,1.6) node (A1) {};
  \draw[->] (Q) -- node[ed] {$B$} +(1.4,-2) node (B1) {};
  \draw[<-] (Q) -- node[ed] {$B$} +(-1.4,-1) node (B2) {};
  \draw[->] (Q) -- node[ed,swap,pos=.3] {$C$} +(-1.3,2.3) node (C1) {};
  %
  \node[draw] (M) at (.5,4) {$M$};
  \node[draw] (N) at (-.5,3) {$N$};
  \node[draw] (P) at (.5,2.5) {$P$};
  \draw[->] (M) -- node[ed] {$D$} (N);
  \draw[->] (N) to[out=90,in=150] node[ed] {$A$} (M);
  \draw[<-] (N) -- node[ed] {$A$} +(-1,-1) node (A2) {};
  \draw[<-] (P) -- node[ed] {$C$} +(-1.7,-1) node (C2) {};
  \draw[<-] (P) -- node[ed,swap,pos=.7] {$D$} +(1,-1.5) node (D1) {};
  \draw[->] (M) -- node[ed] {$B$} +(1.2,.7) node (B3) {};
  \draw[->] (N) to[out=120,in=180,looseness=10] node[ed,swap] {$B$} (N);
  \draw[->] (1.8,4) node (E1) {} to[out=-150,in=-180,looseness=1.5] node[ed,swap] {$E$} +(.2,-.7) node (E2) {};
  %
  \node[draw,isosceles triangle,inner sep=2pt] (r) at (5.5,4.5) {$r$};
  \node[draw,isosceles triangle,inner sep=2pt] (s) at (6,1) {$s$};
  \node[draw,isosceles triangle,inner sep=2pt] (u) at (4,2) {$u$};
  \node[draw,isosceles triangle,inner sep=2pt] (v) at (3.5,5.5) {$v$};
  %
  \begin{scope}[dashed]
    \draw (D1) -- (s) -- (B1);
    \draw (A2) -- (u) -- (B2); \draw (C2) -- (u);
    \draw (B3) -- (r) -- (A1); \draw (E1) -- (r); \draw (E2) -- (r);
    \draw (v) -- (C1);
  \end{scope}
  \node[draw,single arrow] at (5,3.3) {$\alpha$};
\end{tikzc}
For the cell $\alpha$ drawn above:
\begin{itemize}
\item The arrow boundary is $(r,s\op,u,v\op)$, where
  \begin{align*}
    r &: (B\op,E,E\op) \to A\\
    s &: (D\op) \to B\\
    u &: (A,C) \to B\\
    v &: () \to C
  \end{align*}
\item The target is $Q\in\cMod(A,B\op,B,C\op)$.
\item The source is a labeled graph containing the following modules:
  \begin{align*}
    M &\in \cMod(A,B\op,D\op)\\
    N &\in \cMod(A,B,B\op,A\op,D)\\
    P &\in \cMod(C,D)
  \end{align*}
\end{itemize}

\end{document}