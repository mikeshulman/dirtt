\documentclass{amsart}

\newif\ifcref\creftrue
\usepackage{amssymb,amsmath,stmaryrd,mathrsfs}

%% Set this to true before loading if we're using the TAC style file.
%% Note that eventually, TAC requires everything to be in one source file.
\def\definetac{\newif\iftac}    % Can't define a \newif inside another \if!
\ifx\tactrue\undefined
  \definetac
  %% Guess whether we're using TAC by whether \state is defined.
  \ifx\state\undefined\tacfalse\else\tactrue\fi
\fi

% Similarly detect beamer
\def\definebeamer{\newif\ifbeamer}
\ifx\beamertrue\undefined
  \definebeamer
  %% Guess whether we're using Beamer by whether \uncover is defined.
  \ifx\uncover\undefined\beamerfalse\else\beamertrue\fi
\fi

% And cleveref
\def\definecref{\newif\ifcref}
\ifx\creftrue\undefined
  \definecref
  % Default to false
  \creffalse
\fi

\iftac\else\usepackage{amsthm}\fi
\usepackage[all,2cell]{xy}
%\UseAllTwocells
%\usepackage{tikz}
%\usetikzlibrary{arrows}
\ifbeamer\else
  \usepackage{enumitem}
  \usepackage{xcolor}
  \definecolor{darkgreen}{rgb}{0,0.45,0} 
  \usepackage[pagebackref,colorlinks,citecolor=darkgreen,linkcolor=darkgreen]{hyperref}
\fi
\usepackage{mathtools}          % for all sorts of things
\usepackage{graphics}           % for \scalebox, used in \widecheck
\usepackage{ifmtarg}            % used in \jd
\usepackage{microtype}
%\usepackage{color,epsfig}
%\usepackage{fullpage}
%\usepackage{eucal}
%\usepackage{wasysym}
%\usepackage{txfonts}            % for \invamp, or for the nice fonts
\usepackage{braket}             % for \Set, etc.
\let\setof\Set
\usepackage{url}                % for citations to web sites
\usepackage{xspace}             % put spaces after a \command in text
%\usepackage{cite}               % compress and sort grouped citations (only use with numbered citations)
\ifcref\usepackage{cleveref,aliascnt}\fi

%% If you want to use biblatex, e.g. if a journal requires (Author name YEAR) citations.
% \usepackage[style=authoryear,
%  backref=true,
%  maxnames=4,
%  maxbibnames=99,
%  uniquename=false,
%  firstinits=true
% ]{biblatex}
% \addbibresource{all.bib}

% \let\cite\parencite
% \DeclareNameAlias{sortname}{last-first}

\makeatletter
\let\ea\expandafter

%% Defining commands that are always in math mode.
\def\mdef#1#2{\ea\ea\ea\gdef\ea\ea\noexpand#1\ea{\ea\ensuremath\ea{#2}\xspace}}
\def\alwaysmath#1{\ea\ea\ea\global\ea\ea\ea\let\ea\ea\csname your@#1\endcsname\csname #1\endcsname
  \ea\def\csname #1\endcsname{\ensuremath{\csname your@#1\endcsname}\xspace}}

%% WIDECHECK
\DeclareRobustCommand\widecheck[1]{{\mathpalette\@widecheck{#1}}}
\def\@widecheck#1#2{%
    \setbox\z@\hbox{\m@th$#1#2$}%
    \setbox\tw@\hbox{\m@th$#1%
       \widehat{%
          \vrule\@width\z@\@height\ht\z@
          \vrule\@height\z@\@width\wd\z@}$}%
    \dp\tw@-\ht\z@
    \@tempdima\ht\z@ \advance\@tempdima2\ht\tw@ \divide\@tempdima\thr@@
    \setbox\tw@\hbox{%
       \raise\@tempdima\hbox{\scalebox{1}[-1]{\lower\@tempdima\box
\tw@}}}%
    {\ooalign{\box\tw@ \cr \box\z@}}}

%% SIMPLE COMMANDS FOR FONTS AND DECORATIONS

\newcount\foreachcount

\def\foreachletter#1#2#3{\foreachcount=#1
  \ea\loop\ea\ea\ea#3\@alph\foreachcount
  \advance\foreachcount by 1
  \ifnum\foreachcount<#2\repeat}

\def\foreachLetter#1#2#3{\foreachcount=#1
  \ea\loop\ea\ea\ea#3\@Alph\foreachcount
  \advance\foreachcount by 1
  \ifnum\foreachcount<#2\repeat}

% Script: \sA is \mathscr{A}
\def\definescr#1{\ea\gdef\csname s#1\endcsname{\ensuremath{\mathscr{#1}}\xspace}}
\foreachLetter{1}{27}{\definescr}
% Calligraphic: \cA is \mathcal{A}
\def\definecal#1{\ea\gdef\csname c#1\endcsname{\ensuremath{\mathcal{#1}}\xspace}}
\foreachLetter{1}{27}{\definecal}
% Bold: \bA is \mathbf{A}
\def\definebold#1{\ea\gdef\csname b#1\endcsname{\ensuremath{\mathbf{#1}}\xspace}}
\foreachLetter{1}{27}{\definebold}
% Blackboard Bold: \lA is \mathbb{A}
\def\definebb#1{\ea\gdef\csname l#1\endcsname{\ensuremath{\mathbb{#1}}\xspace}}
\foreachLetter{1}{27}{\definebb}
% Fraktur: \ka is \mathfrak{a} (except when it's \kappa, see below), \kA is \mathfrak{A}
\def\definefrak#1{\ea\gdef\csname k#1\endcsname{\ensuremath{\mathfrak{#1}}\xspace}}
\foreachletter{1}{27}{\definefrak}
\foreachLetter{1}{27}{\definefrak}
% Sans serif
\def\definesf#1{\ea\gdef\csname i#1\endcsname{\ensuremath{\mathsf{#1}}\xspace}}
\foreachletter{1}{6}{\definesf}
\foreachletter{7}{14}{\definesf}
\foreachletter{15}{27}{\definesf}
\foreachLetter{1}{27}{\definesf}
% Bar: \Abar is \overline{A}, \abar is \overline{a}
\def\definebar#1{\ea\gdef\csname #1bar\endcsname{\ensuremath{\overline{#1}}\xspace}}
\foreachLetter{1}{27}{\definebar}
\foreachletter{1}{8}{\definebar} % \hbar is something else!
\foreachletter{9}{15}{\definebar} % \obar is something else!
\foreachletter{16}{27}{\definebar}
% Tilde: \Atil is \widetilde{A}, \atil is \widetilde{a}
\def\definetil#1{\ea\gdef\csname #1til\endcsname{\ensuremath{\widetilde{#1}}\xspace}}
\foreachLetter{1}{27}{\definetil}
\foreachletter{1}{27}{\definetil}
% Hats: \Ahat is \widehat{A}, \ahat is \widehat{a}
\def\definehat#1{\ea\gdef\csname #1hat\endcsname{\ensuremath{\widehat{#1}}\xspace}}
\foreachLetter{1}{27}{\definehat}
\foreachletter{1}{27}{\definehat}
% Checks: \Achk is \widecheck{A}, \achk is \widecheck{a}
\def\definechk#1{\ea\gdef\csname #1chk\endcsname{\ensuremath{\widecheck{#1}}\xspace}}
\foreachLetter{1}{27}{\definechk}
\foreachletter{1}{27}{\definechk}
% Underline: \uA is \underline{A}, \ua is \underline{a}
\def\defineul#1{\ea\gdef\csname u#1\endcsname{\ensuremath{\underline{#1}}\xspace}}
\foreachLetter{1}{27}{\defineul}
\foreachletter{1}{27}{\defineul}

% Particular commands for typefaces, sometimes with the first letter
% different.
\def\autofmt@n#1\autofmt@end{\mathrm{#1}}
\def\autofmt@b#1\autofmt@end{\mathbf{#1}}
\def\autofmt@l#1#2\autofmt@end{\mathbb{#1}\mathsf{#2}}
\def\autofmt@c#1#2\autofmt@end{\mathcal{#1}\mathit{#2}}
\def\autofmt@s#1#2\autofmt@end{\mathscr{#1}\mathit{#2}}
\def\autofmt@f#1\autofmt@end{\mathsf{#1}}
\def\autofmt@k#1\autofmt@end{\mathfrak{#1}}
% Particular commands for decorations.
\def\autofmt@u#1\autofmt@end{\underline{\smash{\mathsf{#1}}}}
\def\autofmt@U#1\autofmt@end{\underline{\underline{\smash{\mathsf{#1}}}}}
\def\autofmt@h#1\autofmt@end{\widehat{#1}}
\def\autofmt@r#1\autofmt@end{\overline{#1}}
\def\autofmt@t#1\autofmt@end{\widetilde{#1}}
\def\autofmt@k#1\autofmt@end{\check{#1}}

% Defining multi-letter commands.  Use this like so:
% \autodefs{\bSet\cCat\cCAT\kBicat\lProf}
\def\auto@drop#1{}
\def\autodef#1{\ea\ea\ea\@autodef\ea\ea\ea#1\ea\auto@drop\string#1\autodef@end}
\def\@autodef#1#2#3\autodef@end{%
  \ea\def\ea#1\ea{\ea\ensuremath\ea{\csname autofmt@#2\endcsname#3\autofmt@end}\xspace}}
\def\autodefs@end{blarg!}
\def\autodefs#1{\@autodefs#1\autodefs@end}
\def\@autodefs#1{\ifx#1\autodefs@end%
  \def\autodefs@next{}%
  \else%
  \def\autodefs@next{\autodef#1\@autodefs}%
  \fi\autodefs@next}

%% FONTS AND DECORATION FOR GREEK LETTERS

%% the package `mathbbol' gives us blackboard bold greek and numbers,
%% but it does it by redefining \mathbb to use a different font, so that
%% all the other \mathbb letters look different too.  Here we import the
%% font with bb greek and numbers, but assign it a different name,
%% \mathbbb, so as not to replace the usual one.
\DeclareSymbolFont{bbold}{U}{bbold}{m}{n}
\DeclareSymbolFontAlphabet{\mathbbb}{bbold}
\newcommand{\lDelta}{\ensuremath{\mathbbb{\Delta}}\xspace}
\newcommand{\lone}{\ensuremath{\mathbbb{1}}\xspace}
\newcommand{\ltwo}{\ensuremath{\mathbbb{2}}\xspace}
\newcommand{\lthree}{\ensuremath{\mathbbb{3}}\xspace}

% greek with bars
\newcommand{\albar}{\ensuremath{\overline{\alpha}}\xspace}
\newcommand{\bebar}{\ensuremath{\overline{\beta}}\xspace}
\newcommand{\gmbar}{\ensuremath{\overline{\gamma}}\xspace}
\newcommand{\debar}{\ensuremath{\overline{\delta}}\xspace}
\newcommand{\phibar}{\ensuremath{\overline{\varphi}}\xspace}
\newcommand{\psibar}{\ensuremath{\overline{\psi}}\xspace}
\newcommand{\xibar}{\ensuremath{\overline{\xi}}\xspace}
\newcommand{\ombar}{\ensuremath{\overline{\omega}}\xspace}

% greek with tildes
\newcommand{\altil}{\ensuremath{\widetilde{\alpha}}\xspace}
\newcommand{\betil}{\ensuremath{\widetilde{\beta}}\xspace}
\newcommand{\gmtil}{\ensuremath{\widetilde{\gamma}}\xspace}
\newcommand{\phitil}{\ensuremath{\widetilde{\varphi}}\xspace}
\newcommand{\psitil}{\ensuremath{\widetilde{\psi}}\xspace}
\newcommand{\xitil}{\ensuremath{\widetilde{\xi}}\xspace}
\newcommand{\omtil}{\ensuremath{\widetilde{\omega}}\xspace}

% MISCELLANEOUS SYMBOLS
\let\del\partial
\mdef\delbar{\overline{\partial}}
\let\sm\wedge
\newcommand{\dd}[1]{\ensuremath{\frac{\partial}{\partial {#1}}}}
\newcommand{\inv}{^{-1}}
\newcommand{\dual}{^{\vee}}
\mdef\hf{\textstyle\frac12 }
\mdef\thrd{\textstyle\frac13 }
\mdef\qtr{\textstyle\frac14 }
\let\meet\wedge
\let\join\vee
\let\dn\downarrow
\newcommand{\op}{^{\mathrm{op}}}
\newcommand{\co}{^{\mathrm{co}}}
\newcommand{\coop}{^{\mathrm{coop}}}
\let\adj\dashv
\SelectTips{cm}{}
\newdir{ >}{{}*!/-10pt/\dir{>}}    % extra spacing for tail arrows in XYpic
\makeatother
\newcommand{\pushout}[1][dr]{\save*!/#1+1.2pc/#1:(1,-1)@^{|-}\restore}
\newcommand{\pullback}[1][dr]{\save*!/#1-1.2pc/#1:(-1,1)@^{|-}\restore}
\makeatletter
\let\iso\cong
\let\eqv\simeq
\let\cng\equiv
\mdef\Id{\mathrm{Id}}
\mdef\id{\mathrm{id}}
\alwaysmath{ell}
\alwaysmath{infty}
\let\oo\infty
\alwaysmath{odot}
\def\frc#1/#2.{\frac{#1}{#2}}   % \frc x^2+1 / x^2-1 .
\mdef\ten{\mathrel{\otimes}}
\let\bigten\bigotimes
\mdef\sqten{\mathrel{\boxtimes}}
\def\lt{<}                      % For iTex compatibility
\def\gt{>}

%% OPERATORS
\DeclareMathOperator\lan{Lan}
\DeclareMathOperator\ran{Ran}
\DeclareMathOperator\colim{colim}
\DeclareMathOperator\coeq{coeq}
\DeclareMathOperator\eq{eq}
\DeclareMathOperator\Tot{Tot}
\DeclareMathOperator\cosk{cosk}
\DeclareMathOperator\sk{sk}
%\DeclareMathOperator\im{im}
\DeclareMathOperator\Spec{Spec}
\DeclareMathOperator\Ho{Ho}
\DeclareMathOperator\Aut{Aut}
\DeclareMathOperator\End{End}
\DeclareMathOperator\Hom{Hom}
\DeclareMathOperator\Map{Map}

%% ARROWS
% \to already exists
\newcommand{\too}[1][]{\ensuremath{\overset{#1}{\longrightarrow}}}
\newcommand{\ot}{\ensuremath{\leftarrow}}
\newcommand{\oot}[1][]{\ensuremath{\overset{#1}{\longleftarrow}}}
\let\toot\rightleftarrows
\let\otto\leftrightarrows
\let\Impl\Rightarrow
\let\imp\Rightarrow
\let\toto\rightrightarrows
\let\into\hookrightarrow
\let\xinto\xhookrightarrow
\mdef\we{\overset{\sim}{\longrightarrow}}
\mdef\leftwe{\overset{\sim}{\longleftarrow}}
\let\mono\rightarrowtail
\let\leftmono\leftarrowtail
\let\cof\rightarrowtail
\let\leftcof\leftarrowtail
\let\epi\twoheadrightarrow
\let\leftepi\twoheadleftarrow
\let\fib\twoheadrightarrow
\let\leftfib\twoheadleftarrow
\let\cohto\rightsquigarrow
\let\maps\colon
\newcommand{\spam}{\,:\!}       % \maps for left arrows
\def\acof{\mathrel{\mathrlap{\hspace{3pt}\raisebox{4pt}{$\scriptscriptstyle\sim$}}\mathord{\rightarrowtail}}}

% diagxy redefines \to, along with \toleft, \two, \epi, and \mon.

%% EXTENSIBLE ARROWS
\let\xto\xrightarrow
\let\xot\xleftarrow
% See Voss' Mathmode.tex for instructions on how to create new
% extensible arrows.
\def\rightarrowtailfill@{\arrowfill@{\Yright\joinrel\relbar}\relbar\rightarrow}
\newcommand\xrightarrowtail[2][]{\ext@arrow 0055{\rightarrowtailfill@}{#1}{#2}}
\let\xmono\xrightarrowtail
\let\xcof\xrightarrowtail
\def\twoheadrightarrowfill@{\arrowfill@{\relbar\joinrel\relbar}\relbar\twoheadrightarrow}
\newcommand\xtwoheadrightarrow[2][]{\ext@arrow 0055{\twoheadrightarrowfill@}{#1}{#2}}
\let\xepi\xtwoheadrightarrow
\let\xfib\xtwoheadrightarrow
% Let's leave the left-going ones until I need them.

%% EXTENSIBLE SLASHED ARROWS
% Making extensible slashed arrows, by modifying the underlying AMS code.
% Arguments are:
% 1 = arrowhead on the left (\relbar or \Relbar if none)
% 2 = fill character (usually \relbar or \Relbar)
% 3 = slash character (such as \mapstochar or \Mapstochar)
% 4 = arrowhead on the left (\relbar or \Relbar if none)
% 5 = display mode (\displaystyle etc)
\def\slashedarrowfill@#1#2#3#4#5{%
  $\m@th\thickmuskip0mu\medmuskip\thickmuskip\thinmuskip\thickmuskip
   \relax#5#1\mkern-7mu%
   \cleaders\hbox{$#5\mkern-2mu#2\mkern-2mu$}\hfill
   \mathclap{#3}\mathclap{#2}%
   \cleaders\hbox{$#5\mkern-2mu#2\mkern-2mu$}\hfill
   \mkern-7mu#4$%
}
% Here's the idea: \<slashed>arrowfill@ should be a box containing
% some stretchable space that is the "middle of the arrow".  This
% space is created as a "leader" using \cleader<thing>\hfill, which
% fills an \hfill of space with copies of <thing>.  Here instead of
% just one \cleader, we use two, with the slash in between (and an
% extra copy of the filler, to avoid extra space around the slash).
\def\rightslashedarrowfill@{%
  \slashedarrowfill@\relbar\relbar\mapstochar\rightarrow}
\newcommand\xslashedrightarrow[2][]{%
  \ext@arrow 0055{\rightslashedarrowfill@}{#1}{#2}}
\mdef\hto{\xslashedrightarrow{}}
\mdef\htoo{\xslashedrightarrow{\quad}}
\let\xhto\xslashedrightarrow

%% To get a slashed arrow in XYmatrix, do
% \[\xymatrix{A \ar[r]|-@{|} & B}\]
%% To get it in diagxy, do
% \morphism/{@{>}|-*@{|}}/[A`B;p]

%% Here is an \hto for diagxy:
% \def\htopppp/#1/<#2>^#3_#4{\:%
% \ifnum#2=0%
%    \setwdth{#3}{#4}\deltax=\wdth \divide \deltax by \ul%
%    \advance \deltax by \defaultmargin  \ratchet{\deltax}{100}%
% \else \deltax #2%
% \fi%
% \xy\ar@{#1}|-@{|}^{#3}_{#4}(\deltax,0) \endxy%
% \:}%
% \def\htoppp/#1/<#2>^#3{\ifnextchar_{\htopppp/#1/<#2>^{#3}}{\htopppp/#1/<#2>^{#3}_{}}}%
% \def\htopp/#1/<#2>{\ifnextchar^{\htoppp/#1/<#2>}{\htoppp/#1/<#2>^{}}}%
% \def\htoop/#1/{\ifnextchar<{\htopp/#1/}{\htopp/#1/<0>}}%
% \def\hto{\ifnextchar/{\htoop}{\htoop/>/}}%

% LABELED ISOMORPHISMS
\def\xiso#1{\mathrel{\mathrlap{\smash{\xto[\smash{\raisebox{1.3mm}{$\scriptstyle\sim$}}]{#1}}}\hphantom{\xto{#1}}}}
\def\toiso{\xto{\smash{\raisebox{-.5mm}{$\scriptstyle\sim$}}}}

% SHADOWS
\def\shvar#1#2{{\ensuremath{%
  \hspace{1mm}\makebox[-1mm]{$#1\langle$}\makebox[0mm]{$#1\langle$}\hspace{1mm}%
  {#2}%
  \makebox[1mm]{$#1\rangle$}\makebox[0mm]{$#1\rangle$}%
}}}
\def\sh{\shvar{}}
\def\scriptsh{\shvar{\scriptstyle}}
\def\bigsh{\shvar{\big}}
\def\Bigsh{\shvar{\Big}}
\def\biggsh{\shvar{\bigg}}
\def\Biggsh{\shvar{\Bigg}}

% TYPING JUDGMENTS
% Call this macro as \jd{x:A, y:B |- c:C}.  It adds (what I think is)
% appropriate spacing, plus auto-sized parentheses around each typing judgment.
\def\jd#1{\@jd#1\ej}
\def\@jd#1|-#2\ej{\@@jd#1,,\;\vdash\;\left(#2\right)}
\def\@@jd#1,{\@ifmtarg{#1}{\let\next=\relax}{\left(#1\right)\let\next=\@@@jd}\next}
\def\@@@jd#1,{\@ifmtarg{#1}{\let\next=\relax}{,\,\left(#1\right)\let\next=\@@@jd}\next}
% Here's a version which puts a line break before the turnstyle.
\def\jdm#1{\@jdm#1\ej}
\def\@jdm#1|-#2\ej{\@@jd#1,,\\\vdash\;\left(#2\right)}
% Make an actual comma that doesn't separate typing judgments (e.g. A,B,C : Type).
\def\cm{,}

%% SKIPIT in TikZ
% See http://tex.stackexchange.com/questions/3513/draw-only-some-segments-of-a-path-in-tikz
\long\def\my@drawfill#1#2;{%
\@skipfalse
\fill[#1,draw=none] #2;
\@skiptrue
\draw[#1,fill=none] #2;
}
\newif\if@skip
\newcommand{\skipit}[1]{\if@skip\else#1\fi}
\newcommand{\drawfill}[1][]{\my@drawfill{#1}}

%% TODO: This \autoref in TAC doesn't work with figures (and anything
%% else other than theorems).

%%%% THEOREM-TYPE ENVIRONMENTS, hacked to
%%% (a) number all with the same numbers, and
%%% (b) have the right names.
%% The following code should work in TAC or out of it, and with
%% hyperref or without it.  In all cases, we use \label to label
%% things and \autoref to refer to them (ordinary \ref declines to
%% include names).  The non-hyperref label and reference hack is from
%% Mike Mandell, I believe.
\newif\ifhyperref
\@ifpackageloaded{hyperref}{\hyperreftrue}{\hyperreffalse}
\iftac
  %% In the TAC style, all theorems are actually subsections.  So
  %% the hyperref \autoref doesn't work and we have to use our own code
  %% in any case.  We also have to hook into the \state macros instead
  %% of \@thm since those are what know about the current theorem type.
  \let\your@state\state
  \def\state#1{\my@state#1}
  \def\my@state#1.{\gdef\currthmtype{#1}\your@state{#1.}}
  \let\your@staterm\staterm
  \def\staterm#1{\my@staterm#1}
  \def\my@staterm#1.{\gdef\currthmtype{#1}\your@staterm{#1.}}
  \let\@defthm\newtheorem
  \def\switchtotheoremrm{\let\@defthm\newtheoremrm}
  \def\defthm#1#2#3{\@defthm#1#2} % Ignore the third argument (for cleveref only)
  % Start out \currthmtype as empty
  \def\currthmtype{}
  % In a bit, we're going to redefine \label so that \label{athm} will
  % also make a label "label@name@athm" which is the current value of
  % \currthmtype.  Now \autoref{athm} just adds a reference to this in
  % front of the reference.
  \ifhyperref
    \def\autoref#1{\ref*{label@name@#1}~\ref{#1}}
  \else
    \def\autoref#1{\ref{label@name@#1}~\ref{#1}}
  \fi
  % This has to go AFTER the \begin{document} because apparently
  % hyperref resets the definition of \label at that point.
  \AtBeginDocument{%
    % Save the old definition of \label
    \let\old@label\label%
    % Redefine \label so that \label{athm} will also make a label
    % "label@name@athm" which is the current value of \currthmtype.
    \def\label#1{%
      {\let\your@currentlabel\@currentlabel%
        \edef\@currentlabel{\currthmtype}%
        \old@label{label@name@#1}}%
      \old@label{#1}}
  }
  \let\cref\autoref
\else\ifcref
  % Cleveref does most of it for us.
  \def\defthm#1#2#3{%
    %% Ensure all theorem types are numbered with the same counter
    \newaliascnt{#1}{thm}
    \newtheorem{#1}[#1]{#2}
    \aliascntresetthe{#1}
    %% This command tells cleveref's \cref what to call things
    \crefname{#1}{#2}{#3}% following brace must be on separate line to support poorman cleveref sed file
  }
  % \let\autoref\cref  % May want to use \autoref for xr-ed links
\else
  % In non-TAC styles without cleveref, theorems have their own counters and so the
  % hyperref \autoref works, if hyperref is loaded.
  \ifhyperref
    %% If we have hyperref, then we have to make sure all the theorem
    %% types appear to use different counters so that hyperref can tell
    %% them apart.  However, we want them actually to use the same
    %% counter, so we don't have both Theorem 9.1 and Definition 9.1.
    \def\defthm#1#2#3{% Ignore the third argument (for cleveref only)
      %% All types of theorems are number inside sections
      \newtheorem{#1}{#2}[section]%
      %% This command tells hyperref's \autoref what to call things
      \expandafter\def\csname #1autorefname\endcsname{#2}%
      %% This makes all the theorem counters actually the same counter
      \expandafter\let\csname c@#1\endcsname\c@thm}
  \else
    %% Without hyperref, we have to roll our own.  This code is due to
    %% Mike Mandell.  First, all theorems can now "officially" use the
    %% same counter.
    \def\defthm#1#2#3{\newtheorem{#1}[thm]{#2}} % Ignore the third argument (for cleveref only)
    %% Save the label- and theorem-making commands
    \ifx\SK@label\undefined\let\SK@label\label\fi
    \let\old@label\label
    \let\your@thm\@thm
    %% Save the current type of theorem whenever we start one
    \def\@thm#1#2#3{\gdef\currthmtype{#3}\your@thm{#1}{#2}{#3}}
    %% Start that out as empty
    \def\currthmtype{}
    %% Redefine \label so that \label{athm} defines, in addition to a
    %% label "athm" pointing to "9.1," a label "athm@" pointing to
    %% "Theorem 9.1."
    \def\label#1{{\let\your@currentlabel\@currentlabel\def\@currentlabel%
        {\currthmtype~\your@currentlabel}%
        \SK@label{#1@}}\old@label{#1}}
    %% Now \autoref just looks at "athm@" instead of "athm."
    \def\autoref#1{\ref{#1@}}
  \fi
  \let\cref\autoref
\fi\fi

%% Now the code that works in all cases.  Note that TAC allows the
%% optional arguments, but ignores them.  It also defines environments
%% called "theorem," etc.
\newtheorem{thm}{Theorem}[section]
\ifcref
  \crefname{thm}{Theorem}{Theorems}
\else
  \newcommand{\thmautorefname}{Theorem}
\fi
\defthm{cor}{Corollary}{Corollaries}
\defthm{prop}{Proposition}{Propositions}
\defthm{lem}{Lemma}{Lemmas}
\defthm{sch}{Scholium}{Scholia}
\defthm{assume}{Assumption}{Assumptions}
\defthm{claim}{Claim}{Claims}
\defthm{conj}{Conjecture}{Conjectures}
\defthm{hyp}{Hypothesis}{Hypotheses}
\iftac\switchtotheoremrm\else\theoremstyle{definition}\fi
\defthm{defn}{Definition}{Definitions}
\defthm{notn}{Notation}{Notations}
\iftac\switchtotheoremrm\else\theoremstyle{remark}\fi
\defthm{rmk}{Remark}{Remarks}
\defthm{eg}{Example}{Examples}
\defthm{egs}{Examples}{Examples}
\defthm{ex}{Exercise}{Exercises}
\defthm{ceg}{Counterexample}{Counterexamples}

\ifcref
  % Display format for sections
  \crefformat{section}{\S#2#1#3}
  \Crefformat{section}{Section~#2#1#3}
  \crefrangeformat{section}{\S\S#3#1#4--#5#2#6}
  \Crefrangeformat{section}{Sections~#3#1#4--#5#2#6}
  \crefmultiformat{section}{\S\S#2#1#3}{ and~#2#1#3}{, #2#1#3}{ and~#2#1#3}
  \Crefmultiformat{section}{Sections~#2#1#3}{ and~#2#1#3}{, #2#1#3}{ and~#2#1#3}
  \crefrangemultiformat{section}{\S\S#3#1#4--#5#2#6}{ and~#3#1#4--#5#2#6}{, #3#1#4--#5#2#6}{ and~#3#1#4--#5#2#6}
  \Crefrangemultiformat{section}{Sections~#3#1#4--#5#2#6}{ and~#3#1#4--#5#2#6}{, #3#1#4--#5#2#6}{ and~#3#1#4--#5#2#6}
  % Display format for appendices
  \crefformat{appendix}{Appendix~#2#1#3}
  \Crefformat{appendix}{Appendix~#2#1#3}
  \crefrangeformat{appendix}{Appendices~#3#1#4--#5#2#6}
  \Crefrangeformat{appendix}{Appendices~#3#1#4--#5#2#6}
  \crefmultiformat{appendix}{Appendices~#2#1#3}{ and~#2#1#3}{, #2#1#3}{ and~#2#1#3}
  \Crefmultiformat{appendix}{Appendices~#2#1#3}{ and~#2#1#3}{, #2#1#3}{ and~#2#1#3}
  \crefrangemultiformat{appendix}{Appendices~#3#1#4--#5#2#6}{ and~#3#1#4--#5#2#6}{, #3#1#4--#5#2#6}{ and~#3#1#4--#5#2#6}
  \Crefrangemultiformat{appendix}{Appendices~#3#1#4--#5#2#6}{ and~#3#1#4--#5#2#6}{, #3#1#4--#5#2#6}{ and~#3#1#4--#5#2#6}
  % Display format for parts
  \crefname{part}{Part}{Parts}
  % Display format for figures
  \crefname{figure}{Figure}{Figures}
\fi


% \qedhere for TAC
\iftac
  \let\qed\endproof
  \let\your@endproof\endproof
  \def\my@endproof{\your@endproof}
  \def\endproof{\my@endproof\gdef\my@endproof{\your@endproof}}
  \def\qedhere{\tag*{\endproofbox}\gdef\my@endproof{\relax}}
\fi

% Make the optional arguments to TAC's \proof behave like everyone else's
\iftac
  \def\pr@@f[#1]{\subsubsection*{\sc #1.}}
\fi

% How to get QED symbols inside equations at the end of the statements
% of theorems.  AMS LaTeX knows how to do this inside equations at the
% end of *proofs* with \qedhere, and at the end of the statement of a
% theorem with \qed (meaning no proof will be given), but it can't
% seem to combine the two.  Use this instead.
\def\thmqedhere{\expandafter\csname\csname @currenvir\endcsname @qed\endcsname}

% Number numbered lists as (i), (ii), ...
\ifbeamer\else
  \renewcommand{\theenumi}{(\roman{enumi})}
  \renewcommand{\labelenumi}{\theenumi}
\fi

% Left margins for enumitem
\ifbeamer\else
  \setitemize[1]{leftmargin=2em}
  \setenumerate[1]{leftmargin=*}
\fi

% Also number formulas with the theorem counter
\iftac
  \let\c@equation\c@subsection
\else
  \let\c@equation\c@thm
\fi
\numberwithin{equation}{section}

% Only show numbers for equations that are actually referenced (or
% whose tags are specified manually) - uses mathtools.  All equations
% need to be referenced with \eqref, not \ref, for this to work!
\ifcref\else
  \@ifpackageloaded{mathtools}{\mathtoolsset{showonlyrefs,showmanualtags}}{}
\fi

% GREEK LETTERS, ETC.
\alwaysmath{alpha}
\alwaysmath{beta}
\alwaysmath{gamma}
\alwaysmath{Gamma}
\alwaysmath{delta}
\alwaysmath{Delta}
\alwaysmath{epsilon}
\mdef\ep{\varepsilon}
\alwaysmath{zeta}
\alwaysmath{eta}
\alwaysmath{theta}
\alwaysmath{Theta}
\alwaysmath{iota}
\alwaysmath{kappa}
\alwaysmath{lambda}
\alwaysmath{Lambda}
\alwaysmath{mu}
\alwaysmath{nu}
\alwaysmath{xi}
\alwaysmath{pi}
\alwaysmath{rho}
\alwaysmath{sigma}
\alwaysmath{Sigma}
\alwaysmath{tau}
\alwaysmath{upsilon}
\alwaysmath{Upsilon}
\alwaysmath{phi}
\alwaysmath{Pi}
\alwaysmath{Phi}
\mdef\ph{\varphi}
\alwaysmath{chi}
\alwaysmath{psi}
\alwaysmath{Psi}
\alwaysmath{omega}
\alwaysmath{Omega}
\let\al\alpha
\let\be\beta
\let\gm\gamma
\let\Gm\Gamma
\let\de\delta
\let\De\Delta
\let\si\sigma
\let\Si\Sigma
\let\om\omega
\let\ka\kappa
\let\la\lambda
\let\La\Lambda
\let\ze\zeta
\let\th\theta
\let\Th\Theta
\let\vth\vartheta
\let\Om\Omega

%% Include or exclude solutions
% This code is basically copied from version.sty, except that when the
% solutions are included, we put them in a `proof' environment as
% well.  To include solutions, say \includesolutions; to exclude them
% say \excludesolutions.
% \begingroup
% 
% \catcode`{=12\relax\catcode`}=12\relax%
% \catcode`(=1\relax \catcode`)=2\relax%
% \gdef\includesolutions(\newenvironment(soln)(\begin(proof)[Solution])(\end(proof)))%
% \gdef\excludesolutions(%
%   \gdef\soln(\@bsphack\catcode`{=12\relax\catcode`}=12\relax\soln@NOTE)%
%   \long\gdef\soln@NOTE##1\end{soln}(\solnEND@NOTE)%
%   \gdef\solnEND@NOTE(\@esphack\end(soln))%
% )%
% \endgroup

\makeatother

% Local Variables:
% mode: latex
% TeX-master: ""
% End:

\usepackage{ifmtarg,tikz}
\tikzset{lab/.style={auto,font=\scriptsize}} % arrow labels
\usetikzlibrary{arrows}
\usetikzlibrary{shapes.geometric,shapes.arrows}
\newenvironment{tikzc}[1][]{\begin{center}\begin{tikzpicture}[#1]}{\end{tikzpicture}\end{center}}
\tikzset{>=stealth}
\tikzset{ed/.style={auto,inner sep=0pt,font=\scriptsize}} %edges
\tikzset{arr/.style={draw,isosceles triangle,inner sep=2pt}} %arrows

\newcommand{\C}{\cC}
\newcommand{\V}{\cV}
\newcommand{\W}{\cW}
\newcommand{\K}{\sK}

\autodefs{\cMod}

\newcommand{\blank}{\mathord{\hspace{1pt}\text{--}\hspace{1pt}}}
\newcommand{\uniq}{\mathord{!}}

\title{Cyclic virtual equipments}
\author{Michael Shulman}
\begin{document}
\maketitle

Let $S$ be the monad on $\mathbf{Cat}$ whose algebras are symmetric strict monoidal categories equipped with a symmetric strict monoidal involution.
Concretely, the objects of $S\C$ are finite lists of objects of $\C$, some of which are marked with a formal $(\blank)\op$, such as $(A,B\op,B,C\op,A)$.
The morphisms of $S\C$ are finite lists of morphisms of $\C$ labeled by permutations which respect the opposites, e.g.\ a morphism $(A,B\op,B,C\op,A) \to (D,B,D\op,C,A\op)$ might be given by the following:
\begin{tikzc}
  \node (A1) at (0,3) {$A$};
  \node (Bo2) at (1,3) {$B\op$};
  \node (B3) at (2,3) {$B$};
  \node (Co4) at (3,3) {$C\op$};
  \node (A5) at (4,3) {$A$};
  \node (D1') at (0,0) {$D$};
  \node (B2') at (1,0) {$B$};
  \node (Do3') at (2,0) {$D\op$};
  \node (C4') at (3,0) {$C$};
  \node (Ao5') at (4,0) {$A\op$};
  \draw[->] (A1) to[out=-90,in=90] node[ed,swap,pos=.2] {$f$} (B2');
  \draw[->] (Bo2) to[out=-90,in=90] node[ed,swap,pos=.1] {$g$} (Ao5');
  \draw[->] (B3) to[out=-90,in=90] node[ed,swap,pos=.9] {$h$} (D1');
  \draw[->] (Co4) to[out=-90,in=90] node[ed,swap,pos=.8] {$k$} (Do3');
  \draw[->] (A5) to[out=-90,in=90] node[ed,pos=.2] {$\ell$} (C4');
\end{tikzc}
Here $f:A\to B$, $g:B\to A$, $h:B\to D$, $k:C\to D$, and $\ell:A\to C$ are morphisms in $\C$.
Note that there can only be a morphism between two lists if they have the same length \emph{and} the same number of opposites.

Since $S\C$ is monoidal with an involution, its objects can be concatenated and oppositized, so for instance
\[(A,B\op)\cdot (C\op,D,A) = (A,B\op,C\op,D,A) \quad\text{and}\quad (B\op,A,A)\op = (B,A\op,A\op).\]
Now suppose $\C$ is a groupoid.
If $\alpha$ is a word in $S\C$ of length $n$, define a \emph{pairing} on $\alpha$ to be a partition of $[n]$ into 2-element subsets (so that in particular $n$ must be even) such that in each pair exactly one of the corresponding objects in $\alpha$ is opposite (so that in particular exactly half of the objects in $\alpha$ must be opposite), together with isomorphisms in $\C$ between each two paired objects.
Now let $H_\C$ denote the \emph{pairing profunctor} from $S\C$ to $S\C$, where $H_\C(\alpha,\beta)$ is the set of pairings on $\alpha\op\cdot\beta$.
Thus we either pair two objects in $\alpha$ or two objects in $\beta$ of which one is opposite, or we pair an object in $\alpha$ with an object in $\beta$ that are both or neither opposite.
For instance, an element of $H_\C((A,B\op,A\op,C),(C,B\op)$ could be drawn like this:
\begin{tikzc}[scale=.7]
  \node (A1) at (0,3) {$A$};
  \node (Bo2) at (0,2) {$B\op$};
  \node (Ao3) at (0,1) {$A\op$};
  \node (C4) at (0,0) {$C$};
  \node (C1') at (3,1) {$C$};
  \node (Bo2') at (3,0) {$B\op$};
  \node (D3') at (3,2) {$D$};
  \node (D4') at (3,3) {$D\op$};
  \draw (A1) to[out=0,in=0] (Ao3);
  \draw (Bo2) to[out=0,in=180] (Bo2');
  \draw (C4) to[out=0,in=180] (C1');
  \draw (D3') to[out=180,in=180,looseness=1.5] (D4');
\end{tikzc}
The actions of $S\C$ are given by rearranging the pairs and composing isomorphisms; we need $\C$ to be a groupoid since we end up having to compose in both directions.

In what follows we will be interested in a particular restriction of the pairing profunctor.
Let $\mu:SS\to S$ denote the multiplication of the monad $S$.
Let $R$ denote the free strict monoidal category monad, which comes with an obvious map of monads $\theta : R\to S$.
Define
\begin{align*}
  H'_X &= H_X(\mu \circ \theta S,\mu)\\
  &\cong \mu_* \odot H_X \odot \mu^* \odot \theta S^*
\end{align*}
This is a profunctor from $SSX$ to $RSX$; thus we have just decomposed the domain and codomain of $H$ in a particular way.

Now we define a \textbf{node family} to consist of a groupoid $X$ together with a diagram $Y:SX\to\mathbf{Set}$, which we will often regard as a profunctor from $SX$ to $1$.
We picture the elements of $Y$ as labels for nodes in directed graphs, equipped with an (ordered) labeling on their edges: the objects labeled with $(\blank)\op$ correspond to edges going out, while those without $(\blank)\op$ are edges going in.
For instance, $M\in Y(A,B\op,B,C\op,A)$ might be drawn like this:
\begin{tikzc}
  \node[rectangle,draw] (M) at (0,0) {$M$};
  \draw[<-] (M) -- node[ed] {$A$} (1,1) node[ed,anchor=north west] {1};
  \draw[->] (M) -- node[ed] {$B$} (1,-1) node[ed,anchor=south west] {2};
  \draw[<-] (M) -- node[ed] {$B$} (-.3,-1) node[ed,anchor=north east] {3};
  \draw[->] (M) -- node[ed] {$C$} (-1,0) node[ed,anchor=east] {4};
  \draw[<-] (M) -- node[ed] {$A$} (-.3,1) node[ed,anchor=south east] {5};
\end{tikzc}
The morphisms in $SX$ are permutations labeled by isomorphisms in $X$ that respect the opposites.
This essentially means that the morphisms of $X$ act on nodes, and moreover we have operations allowing us to renumber the edges of any node.
Note that these actions are not in general free: if for instance $M \in Y(A,A)$ it might, or might not, be the case that $M$ is fixed by switching the two $A$s.

Given a node family $(X,Y)$, we define a new node family $T(X,Y)$ as follows.
Its underlying groupoid is $S X$.
To define the nodes of $T(X,Y)$, first note that the free strict monoidal category monad $R$ can be extended to act on profunctors, giving a profunctor $RY : RSX \to R1$.
The elements of $RY(n,\omega)$, where $\omega$ is a list of elements of $SX$ of length $n$, are permutations of $n$ labeled by elements of $Y$.
In other words, $RY(n,\omega)$ is a copower by $\Sigma_n$ of the set of length-$n$ lists of elements of $Y$.
Thus, if we left Kan extend $RY$ along the unique functor $\uniq:R1\to 1$ (which is the same as composing it with the representable profunctor ``$\uniq_*$'' from $R1$ to $1$), we get a profunctor whose elements are precisely finite lists of elements of $Y$.
Now we also compose this profunctor on the left with $H'_X$, obtaining a profunctor
\[ H'_X \odot RY \odot \uniq_* : SSX \hto 1.\]
This is the diagram of nodes in $T(X,Y)$.
More concretely, they are obtained as follows.
\begin{enumerate}
\item Consider a finite list of nodes in $Y$
\item Renumber their combined edges arbitrarily (compatibly with how we already know how to renumber the edges of each node individually)
\item Pair up some of their edges that have opposite arity (the loops on the left side of $H$) and perhaps add some new ``free'' edges having no nodes (the loops on the right side of $H$)
\item Decompose the labels on the remaining outer edges into a word of words for $S$, e.g.\ $(A,B\op,C,B,B\op,A\op)$ could become $((A,B\op),(C),(),(B\op,B,A)\op)$.
\end{enumerate}
We call these nodes of $T(X,Y)$ \emph{labeled graphs}.
If the nodes are ``generalized proarrows'', then a labeled graph is a way to ``compose them up'': the pairings indicate where to perform tensor products of functors, the disconnected components of the graph are simply tensored together, and the free edges added on the right of $H$ are hom-functors to also tensor in.

The operation $T$ is almost a monad on the category of node families.
We have a map $(X,Y) \to T(X,Y)$ arising from the units of the monads $S$ and $R$: on nodes it does no renumbering, pairing, or antipairing, and decomposes $(A,B\op,C)$ as $((A),(B)\op,(C))$.
And we have a \emph{partial} map $TT(X,Y)\rightharpoonup T(X,Y)$ defined as follows.

On objects it is simply the multiplication of the monad $S$ (and hence is total).
On nodes, what we need is a map
\[H'_{S X} \odot R(H'_X \odot RY \odot \uniq_*) \odot \uniq_* \rightharpoonup H'_X \odot RY \odot \uniq_*\]
lying over the map $S\mu : SSS X \to SS X$ on the left.
Using the fact that $R$ extended to profunctors is strong, the desired domain can be identified with
\[ H'_{S X} \odot RH'_X \odot RRY \odot R(\uniq_*) \odot \uniq_* \]
Expanding out the definition of $H'$ in terms of $H$, the desired map is shown in \cref{fig:Tmult}.
\begin{figure}
  \centering
  \newcounter{reg}
  \def\labreg#1{\stepcounter{reg}(\Alph{reg})%
    \ea\global\ea\edef\csname refreg#1\endcsname{(\Alph{reg})}}
  \begin{equation*}
    \vcenter{\xymatrix{
        \ar[r]^{\mu S_*}\ar[ddd]_{S\mu } &
        \ar[r]^{H_{SX}}\ar[dd]_\mu \ar@{}[ddr]|{\Downarrow\labreg{HSX}} &
        \ar[r]^{\mu S^*}\ar[dd]^\mu &
        \ar[r]^{\theta SS^*} &
        \ar[r]^{R\mu_*} &
        \ar[r]^{RH_X}\ar[d]_{\theta S} \ar@{}[dr]|{\Downarrow\theta} &
        \ar[r]^{R\mu^*}\ar[d]^{\theta S} &
        \ar[r]^{R\theta S^*} &
        \ar[r]^{RRY}\ar[ddd]_{\nu} \ar@{}[dddr]|{\Downarrow\nu} &
        \ar[r]^{R\uniq_*}\ar[ddd]^{\nu} &
        \ar[r]^{\uniq_*} &
        \ar@{=}[ddd]\\
        &&&&&
        \ar[r]|{S H_X}\ar[d]_{\mu} \ar@{}[dr]|{\Downarrow\labreg{SHX}} &
        \ar[d]^{\mu} &&&&
        \\
        &
        \ar[r]_{H_X}\ar@{=}[d] &
        \ar@{=}[rrr] \ar@{}[drrr]|{\Downarrow\labreg{HH}} &&&
        \ar[r]_{H_X} &
        \ar@{=}[d] &&&&
        \\
        \ar[r]_{\mu_*} &
        \ar[rrrrr]_{H_X} &&&&&
        \ar[r]_{\mu^*} &
        \ar[r]_{\theta S^*} &
        \ar[r]_{RY} &
        \ar[rr]_{\uniq_*}&&
      }}
  \end{equation*}
  \caption{The multiplication of $T$}
  \label{fig:Tmult}
\end{figure}
The empty regions all arise from commutative diagrams of functors.
The regions marked $\theta$ and $\nu$ are just the extension of those transformations to maps on profunctors.
Region \refregHSX\ ``distributes'' a pairing between words to a pairing between their elements, while region \refregSHX\ discards the brackets in a list of pairings.

It remains, therefore, to consider region \refregHH, and here is where the partiality enters.
We define this partial composition law $H\otimes H \rightharpoonup H$ by ``following the paths of pairings'' until they end up on the outside, as below:
 \begin{tikzc}[scale=.7]
  \node (A1) at (0,3) {$A$};
  \node (Bo2) at (0,2) {$B\op$};
  \node (Ao3) at (0,1) {$A\op$};
  \node (C4) at (0,0) {$C$};
  \node (C1') at (3,1) {$C$};
  \node (Bo2') at (3,0) {$B\op$};
  \node (D3') at (3,2) {$C\op$};
  \node (D4') at (3,3) {$C$};
  \node (Do1'') at (6,2) {$C$};
  \node (Bo2'') at (6,1) {$B\op$};
  \draw (A1) to[out=0,in=0] (Ao3);
  \draw (Bo2) to[out=0,in=180] (Bo2');
  \draw (C4) to[out=0,in=180] (C1');
  \draw (D3') to[out=180,in=180,looseness=1.5] (D4');
  \draw (D4') to[out=0,in=180] (Do1'');
  \draw (Bo2') to[out=0,in=180] (Bo2'');
  \draw (C1') to[out=0,in=0,looseness=1.5] (D3');
  \node at (8,1.5) {$\Rightarrow$};
  \begin{scope}[xshift=10cm]
    \node (xA1) at (0,3) {$A$};
    \node (xBo2) at (0,2) {$B\op$};
    \node (xAo3) at (0,1) {$A\op$};
    \node (xC4) at (0,0) {$C$};
    \node (xDo1'') at (3,2) {$C$};
    \node (xBo2'') at (3,1) {$B\op$};
    \draw (xA1) to[out=0,in=0] (xAo3);
    \draw (xBo2) to[out=0,in=180] (xBo2'');
    \draw (xC4) to[out=0,in=180] (xDo1'');
  \end{scope}
\end{tikzc}
However, this composition is only defined when there are no resulting ``loops'' in the middle:
\begin{tikzc}
  \matrix[column sep=1.2cm,row sep=.3cm]{\node (a11) {$A$}; & \node (a21) {$A$}; & \\
    \node (a12) {$C\op$}; & \node (a22) {$D\op$}; & \node (a32) {$A$}; \\
    \node (a13) {$B$}; & \node (a23) {$D$}; & \node (a33) {$B$}; \\
    \node (a14) {$C$}; & \node (a24) {$B$}; & \\
  };
  \draw (a11) -- (a21) to[out=0,in=180] (a32);
  \draw (a13) to[out=0,in=180] (a24);
  \draw (a24) to[out=0,in=180] (a33);
  \draw (a12) to[out=0,in=0] (a14);
  \draw (a22) to[out=0,in=0,looseness=1.5] (a23);
  \draw (a22) to[out=180,in=180,looseness=1.5] (a23);
  \node[red!40,rotate=20] {\LARGE NOT ALLOWED};
\end{tikzc}
In other words, while we allow ourselves to create ``loops'' by pairing up two edges of the same node, or more generally by creating circuits between nodes with the pairings, we do not allow ``free loops'' that contain no nodes at all.

It is easy to see that both composites $T(X,Y) \to TT(X,Y) \rightharpoonup T(X,Y)$ are (total and) the identity.
The associativity square is more tedious but still essentially straightforward.
Moreover, insofar as it makes sense, the ``monad'' $T$ is \emph{cartesian}, enabling us to apply Leinster's machinery of generalized multicategories to it.\footnote{The reason we have to forbid free loops is to obtain both the associativity square and the cartesian property: if we \emph{list} the free loops in some order, then the associativity square will only hold ``up to isomorphism'' since these lists may have to be reordered, whereas if we give them no order at all (such as considering the free abelian group on the set of objects) then cartesianness would fail.}
% TODO: Check those more carefully.

A $T$-multicategory, therefore, consists of a span $(X_0,Y_0) \leftarrow (X_1,Y_1) \to T(X_0,Y_0)$ of node families, with unit and compositional structure.
We deal with the partiality of the monad multiplication by only requiring the multicategory composition to be defined when the multiplication of $T$ is.
We further restrict our $T$-multicategories by asking that $X_0$ be a discrete set, and that the functor $X_1 \to S X_0$ be a discrete fibration (thus $X_1$ is not just a span but a profunctor from $X_0$ to $S X_0$).
A $T$-multicategory with these properties we will call a \textbf{virtual rigid double category}.

We call $X_0$ the \textbf{objects} and $X_1$ the \textbf{arrows}.
They form an $S$-multicategory; this is like an ordinary symmetric multicategory, but some of the objects in the domain of an arrow can be labeled ``opposite'', and such opposites ``distribute through'' when morphisms are composed.
Of course, $(X_0,Y_0)$ is a node family; we call the elements of $Y_0$ \textbf{modules} and their parametrizing word in $SX_0$ their \textbf{boundary}; if $\omega\in SX_0$ we write $\cMod(\omega)$ for the set (and, later, the category) of modules with boundary $\omega$.

It remains to describe $Y_1$, whose elements we call \textbf{cells}.
This is a node family over $X_1$, the set of morphisms in the underlying $S$-multicategory; thus each of its elements is parametrized by a finite list of arrows, some opposite.
We call this list its \textbf{arrow boundary}.
The \textbf{target} map of the $T$-multicategory assigns to any cell $\alpha$ a module whose boundary is the target of the arrow boundary of $\alpha$.

Finally, the \textbf{source} of a cell $\alpha$ is a labeled graph over $(X_0,Y_0)$ (a diagram of modules that ``could be composed'').
Recalling that the outer edges of a labeled graph are decomposed into a word of words, this must match up with the word obtained from the sources of the arrow boundary of $\alpha$.

\begin{figure}
  \centering
  \begin{tikzc}[xscale=.9]
  \node[draw] (Q) at (10,3) {$Q$};
  \draw[<-] (Q) -- node[ed,swap] {$A$} +(1.5,1.6) node (A1) {};
  \draw[->] (Q) -- node[ed] {$B$} +(1.4,-2) node (B1) {};
  \draw[<-] (Q) -- node[ed] {$B$} +(-1.4,-1) node (B2) {};
  \draw[->] (Q) -- node[ed,swap,pos=.3] {$C$} +(-1.3,2.3) node (C1) {};
  %
  \node[draw] (M) at (.5,4) {$M$};
  \node[draw] (N) at (-.5,3) {$N$};
  \node[draw] (P) at (.5,2.5) {$P$};
  \draw[->] (M) -- node[ed] {$D$} (N);
  \draw[->] (N) to[out=90,in=150] node[ed] {$A$} (M);
  \draw[<-] (N) -- node[ed] {$A$} +(-1,-1) node (A2) {};
  \draw[<-] (P) -- node[ed] {$C$} +(-1.7,-1) node (C2) {};
  \draw[<-] (P) -- node[ed,swap,pos=.7] {$D$} +(1,-1.5) node (D1) {};
  \draw[->] (M) -- node[ed] {$B$} +(1.2,.7) node (B3) {};
  \draw[->] (N) to[out=120,in=180,looseness=10] node[ed,swap] {$B$} (N);
  \draw[->] (1.8,4) node (E1) {} to[out=-150,in=-180,looseness=1.5] node[ed,swap] {$E$} +(.2,-.7) node (E2) {};
  %
  \node[arr] (r) at (5.5,4.5) {$r$};
  \node[arr] (s) at (6,1) {$s$};
  \node[arr] (u) at (4,2) {$u$};
  \node[arr] (v) at (3.5,5.5) {$v$};
  %
  \begin{scope}[dashed]
    \draw (D1) -- (s) -- (B1);
    \draw (A2) -- (u) -- (B2); \draw (C2) -- (u);
    \draw (B3) -- (r) -- (A1); \draw (E1) -- (r); \draw (E2) -- (r);
    \draw (v) -- (C1);
  \end{scope}
  \node[draw,single arrow] at (5,3.3) {$\alpha$};
\end{tikzc}
\caption{A cell in a virtual rigid double category}
\label{fig:genericcell}
\end{figure}

In \cref{fig:genericcell} is drawn a generic cell $\alpha$ with its source, target, and arrow boundary.
We have not numbered anything on the picture, but technically all the arrows, nodes, and edges appear in ordered lists.
For this cell $\alpha$:
\begin{itemize}
\item The arrow boundary is $(r,s\op,u,v\op)$, where
  \begin{align*}
    r &: (B\op,E,E\op) \to A\\
    s &: (D\op) \to B\\
    u &: (A,C) \to B\\
    v &: () \to C
  \end{align*}
\item The target is $Q\in\cMod(A,B\op,B,C\op)$.
\item The source is a labeled graph containing the following modules:
  \begin{align*}
    M &\in \cMod(A,B\op,D\op)\\
    N &\in \cMod(A,B,B\op,A\op,D)\\
    P &\in \cMod(C,D)
  \end{align*}
\end{itemize}
The composition of cells is hard to draw, but not as hard to imagine: for each module $M_i$ in the source of a given cell $\alpha$, we specify another cell $\beta_i$ having $M_i$ as its target, and we require that the arrow boundaries of the $\beta$'s match up wherever the $M$'s are paired together in the source of $\alpha$.
(More precisely, the $\beta$'s form a labeled graph in $Y_1$.)
Then we have a composite $\alpha(\beta_1,\dots,\beta_n)$, whose target is the target of $\alpha$, and whose source is obtained by ``grafting'' the source of each $\beta_i$ at the place where $M_i$ appears in the source of $\alpha$.
Its arrow boundary is obtained from the multicategorical composition in the $S$-multicategory of arrows.

One potentially confusing special case is when the source of $\alpha$ contains \emph{no} modules, i.e.\ it consists only of some collection of free edges.
In this case, the labeled graph in $Y_1$ that we compose with also contains no actual cells $\beta$, but it can still be nontrivial: it consists of an analogous collection of free edges on \emph{arrows}.

It is also worth considering some degenerate cases.
For instance, the modules with fixed boundary $\omega$ form a category, also denoted $\cMod(\omega)$, in which the morphisms are cells whose arrow boundary consists only of identities.

\begin{figure}
  \centering
  \begin{tikzc}[yscale=1.2]
    \node[draw] (Q) at (6,2) {$N$};
    \node[draw] (M) at (0,1) {$M_1$};
    \node[draw] (N) at (0,2) {$M_2$};
    \node[draw] (P) at (0,3) {$M_3$};
    \node (A) at (0,0) {};
    \node (D) at (0,4) {};
    \draw[->] (A) -- node[ed] {$A_0$} (M);
    \draw[->] (M) -- node[ed] {$A_1$} (N);
    \draw[->] (N) -- node[ed] {$A_2$} (P);
    \draw[->] (P) -- node[ed] {$A_3$} (D);
    \node (E) at (6,0) {};
    \node (F) at (6,4) {};
    \draw[->] (E) -- node[ed] {$B$} (Q);
    \draw[->] (Q) -- node[ed] {$C$} (F);
    \node[arr] (f) at (3,4) {$u$};
    \node[arr] (g) at (3,0) {$v$};
    \draw[dashed] (A) -- (g) -- (E);
    \draw[dashed] (D) -- (f) -- (F);
    \node[draw,single arrow] at (3,2) {$\alpha$};
  \end{tikzc}
  \caption{A virtual double category cell}
  \label{fig:vdc}
\end{figure}

On the other hand, if we consider only arrows with unary source $(A)$, and only modules with boundary of the form $(A,B\op)$, then our cells look like the one shown in \cref{fig:vdc}.
%Note that the arrow boundary is $(u\op,v)$.
In this way, every virtual rigid double category has an underlying ordinary virtual double category.

If we retain the restriction that modules have boundary $(A,B\op)$, but allow arrows with arbitrary sources containing no opposites, then we obtain cells belonging to an (as yet undefined) notion that one might call a ``virtual monoidal double category''.
Such a cell is shown in \cref{fig:vmdc}.

\begin{figure}
  \centering
  \begin{tikzc}[yscale=1.2]
    \node[draw] (Q) at (6,2) {$Q$};
    \begin{scope}
      \node[draw] (M) at (0,1) {$N_1$};
      \node[draw] (N) at (0,2) {$N_2$};
      \node[draw] (P) at (0,3) {$N_3$};
      \node (A) at (0,0) {};
      \node (D) at (0,4) {};
      \draw[->] (A) -- node[ed] {$B_0$} (M);
      \draw[->] (M) -- node[ed] {$B_1$} (N);
      \draw[->] (N) -- node[ed] {$B_2$} (P);
      \draw[->] (P) -- node[ed] {$B_3$} (D);
    \end{scope}
    \begin{scope}[xshift=1cm,yshift=-.5cm]
      \node[draw] (M') at (0,2) {$M$};
      \node (A') at (0,0) {};
      \node (D') at (0,4) {};
      \draw[->] (A') -- node[ed,swap] {$A_0$} (M');
      \draw[->] (M') -- node[ed,swap] {$A_1$} (D');
    \end{scope}
    \begin{scope}[xshift=-1cm,yshift=.5cm]
      \node[draw] (M'') at (0,1.3) {$P_1$};
      \node[draw] (N'') at (0,2.7) {$P_2$};
      \node (A'') at (0,0) {};
      \node (D'') at (0,4) {};
      \draw[->] (A'') -- node[ed] {$C_0$} (M'');
      \draw[->] (M'') -- node[ed] {$C_1$} (N'');
      \draw[->] (N'') -- node[ed] {$C_2$} (D'');
    \end{scope}
    \node (E) at (6,0) {};
    \node (F) at (6,4) {};
    \draw[->] (E) -- node[ed] {$D$} (Q);
    \draw[->] (Q) -- node[ed] {$E$} (F);
    \node[arr] (f) at (3,4) {$u$};
    \node[arr] (g) at (3,0) {$v$};
    \draw[dashed] (A) -- (g) -- (E); \draw[dashed] (A') -- (g); \draw[dashed] (A'') -- (g);
    \draw[dashed] (D) -- (f) -- (F); \draw[dashed] (D') -- (f); \draw[dashed] (D'') -- (f);
    \node[draw,single arrow] at (3,2) {$\alpha$};
  \end{tikzc}
  \caption{A virtual monoidal double category cell}
  \label{fig:vmdc}
\end{figure}

\end{document}